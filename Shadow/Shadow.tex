\documentclass[journal,12pt,twocolumn]{IEEEtran}
 \usepackage{setspace}
 \usepackage{gensymb}
 \usepackage{graphicx}
 \singlespacing
\graphicspath{ {/user/adarshsrivastava/desktop/Matrix Theory/Assignemnt_7} }
 \usepackage[cmex10]{amsmath}
 \usepackage{amsthm}
 \usepackage{hyperref}
 \usepackage{mathrsfs}
 \usepackage{txfonts}
 \usepackage{stfloats}
 \usepackage{bm}
 \usepackage{cite}
 \usepackage{cases}
 \usepackage{subfig}
 \usepackage{longtable}
 \usepackage{multirow}
 \usepackage{enumitem}
 \usepackage{mathtools}
 \usepackage{steinmetz}
 \usepackage{tikz}
 \usepackage{circuitikz}
 \usepackage{verbatim}
 \usepackage{tfrupee}
 \usepackage[breaklinks=true]{hyperref}
 \usepackage{tkz-euclide}
 \usetikzlibrary{calc,math}
 \usepackage{listings}
     \usepackage{color}                                            %%
     \usepackage{array}                                            %%
     \usepackage{longtable}                                        %%
     \usepackage{calc}                                             %%
     \usepackage{multirow}                                         %%
     \usepackage{hhline}                                           %%
     \usepackage{ifthen}                                           %%
     \usepackage{lscape}     
 \usepackage{multicol}
 \usepackage{chngcntr}
 \DeclareMathOperator*{\Res}{Res}
 \renewcommand\thesection{\arabic{section}}
 \renewcommand\thesubsection{\thesection.\arabic{subsection}}
 \renewcommand\thesubsubsection{\thesubsection.\arabic{subsubsection}}
 \renewcommand\thesectiondis{\arabic{section}}
 \renewcommand\thesubsectiondis{\thesectiondis.\arabic{subsection}}
 \renewcommand\thesubsubsectiondis{\thesubsectiondis.\arabic{subsubsection}}
 \hyphenation{op-tical net-works semi-conduc-tor}
 \def\inputGnumericTable{}                                 %%
 \lstset{
 %language=C,
 frame=single, 
 breaklines=true,
 columns=fullflexible
 }
 \begin{document}
 \newtheorem{theorem}{Theorem}[section]
 \newtheorem{problem}{Problem}
 \newtheorem{proposition}{Proposition}[section]
 \newtheorem{lemma}{Lemma}[section]
 \newtheorem{corollary}[theorem]{Corollary}
 \newtheorem{example}{Example}[section]
 \newtheorem{definition}[problem]{Definition}
 \newcommand{\BEQA}{\begin{eqnarray}}
 \newcommand{\EEQA}{\end{eqnarray}}
 \newcommand{\define}{\stackrel{\triangle}{=}}
 \bibliographystyle{IEEEtran}
 \providecommand{\mbf}{\mathbf}
 \providecommand{\pr}[1]{\ensuremath{\Pr\left(#1\right)}}
 \providecommand{\qfunc}[1]{\ensuremath{Q\left(#1\right)}}
 \providecommand{\sbrak}[1]{\ensuremath{{}\left[#1\right]}}
 \providecommand{\lsbrak}[1]{\ensuremath{{}\left[#1\right.}}
 \providecommand{\rsbrak}[1]{\ensuremath{{}\left.#1\right]}}
 \providecommand{\brak}[1]{\ensuremath{\left(#1\right)}}
 \providecommand{\lbrak}[1]{\ensuremath{\left(#1\right.}}
 \providecommand{\rbrak}[1]{\ensuremath{\left.#1\right)}}
 \providecommand{\cbrak}[1]{\ensuremath{\left\{#1\right\}}}
 \providecommand{\lcbrak}[1]{\ensuremath{\left\{#1\right.}}
 \providecommand{\rcbrak}[1]{\ensuremath{\left.#1\right\}}}
 \theoremstyle{remark}
 \newtheorem{rem}{Remark}
 \newcommand{\sgn}{\mathop{\mathrm{sgn}}}
 \providecommand{\abs}[1]{\left\vert#1\right\vert}
 \providecommand{\res}[1]{\Res\displaylimits_{#1}} 
 \providecommand{\norm}[1]{\left\lVert#1\right\rVert}
 %\providecommand{\norm}[1]{\lVert#1\rVert}
 \providecommand{\mtx}[1]{\mathbf{#1}}
 \providecommand{\mean}[1]{E\left[ #1 \right]}
 \providecommand{\fourier}{\overset{\mathcal{F}}{ \rightleftharpoons}}
 %\providecommand{\hilbert}{\overset{\mathcal{H}}{ \rightleftharpoons}}
 \providecommand{\system}{\overset{\mathcal{H}}{ \longleftrightarrow}}
 	%\newcommand{\solution}[2]{\textbf{Solution:}{#1}}
 \newcommand{\solution}{\noindent \textbf{Solution: }}
 \newcommand{\cosec}{\,\text{cosec}\,}
 \providecommand{\dec}[2]{\ensuremath{\overset{#1}{\underset{#2}{\gtrless}}}}
 \newcommand{\myvec}[1]{\ensuremath{\begin{pmatrix}#1\end{pmatrix}}}
 \newcommand{\mydet}[1]{\ensuremath{\begin{vmatrix}#1\end{vmatrix}}}
 \numberwithin{equation}{subsection}
 \makeatletter
 \@addtoreset{figure}{problem}
 \makeatother
 \let\StandardTheFigure\thefigure
 \let\vec\mathbf
 \renewcommand{\thefigure}{\theproblem}
 \def\putbox#1#2#3{\makebox[0in][l]{\makebox[#1][l]{}\raisebox{\baselineskip}[0in][0in]{\raisebox{#2}[0in][0in]{#3}}}}
      \def\rightbox#1{\makebox[0in][r]{#1}}
      \def\centbox#1{\makebox[0in]{#1}}
      \def\topbox#1{\raisebox{-\baselineskip}[0in][0in]{#1}}
      \def\midbox#1{\raisebox{-0.5\baselineskip}[0in][0in]{#1}}
 \vspace{3cm}
 \title{Shadow Assignment}
 \author{Adarsh Srivastava}
 \maketitle
 \newpage
 \bigskip
 %\renewcommand{\thefigure}{\theenumi}
 \renewcommand{\thetable}{\theenumi}
 The link to the solution is
 \begin{lstlisting}
  https://github.com/Adarsh1310/EE5609
 \end{lstlisting}
 \begin{abstract}
 This documents finds shadow of an object on a plane.
 \end{abstract}
  \section{\textbf{Problem}}
Find the shadow of an object on the plane described by the given orthonormal vectors.
\section{\textbf{Solution}}
Let $\vec{O}$ be the plane whose projection has to be obtained and \vec{P} be the final projection on the given plane.We have been provided with two orthonormal vector, let us call them $\vec{a}_1$ and $\vec{a}_2$.Using these vectors we can find the plane as,
\begin{align}
\vec{A}=\myvec{\vec{a}_1&\vec{a}_2}
\end{align}
Let,
\begin{align}
\vec{e}=\vec{O}-\vec{P}\label{e}
\end{align}
This vector will be perpendicular to the $\vec{P}$.\\
Now,$\vec{P}$ is just some linear combinations of $\vec{a}_1$, $\vec{a}_2$ and can be written as
\begin{align}
\vec{P}=x_1\vec{a}_1+x_2\vec{a}_2\\
=\vec{A}\vec{x}\label{p}
\end{align}
Our goal is to find $\vec{x}$.We know that the vector $\vec{e}$ obtained in \eqref{e} is perpendicular to the planes.Hence,
\begin{align}
{\vec{a}_1}^T(\vec{O}-\vec{A}\vec{x})=0\\
{\vec{a}_2}^T(\vec{O}-\vec{A}\vec{x})=0\\
\myvec{{a_1}\\{a_2}}(\vec{O}-\vec{A}\vec{x})=\myvec{0\\0}\\
\vec{A}^T(\vec{O}-\vec{A}\vec{x})=0\label{a}
\end{align}
Form \eqref{a} we can see that $\vec{e}$ is in the nullspace of $\vec{A}^T$ which means that $\vec{e}$ is perpendicular to column space of $\vec{A}^T$.
\begin{align}
\vec{A}^T\vec{A}\vec{x}=\vec{A}^T\vec{O}
\end{align}
From equation \eqref{a} we can find the value of \vec{x}
\begin{align}
\vec{x}=(\vec{A}^T\vec{A})^{-1}\vec{A}^T\vec{O}\label{x}
\end{align}
Substituting \eqref{x} in \eqref{p}
\begin{align}
\vec{P}=\vec{A}(\vec{A}^T\vec{A})^{-1}\vec{A}^T\vec{O}
\end{align}
Hence, We can find the projection vector by multiplying $\vec{A}(\vec{A}^T\vec{A})^{-1}\vec{A}^T$ to the given space $\vec{O}$
 \end{document}