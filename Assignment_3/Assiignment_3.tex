\documentclass[journal,12pt,twocolumn]{IEEEtran}

\usepackage{setspace}
\usepackage{gensymb}

\singlespacing


\usepackage[cmex10]{amsmath}

\usepackage{amsthm}
\usepackage{hyperref}
\usepackage{mathrsfs}
\usepackage{txfonts}
\usepackage{stfloats}
\usepackage{bm}
\usepackage{cite}
\usepackage{cases}
\usepackage{subfig}

\usepackage{longtable}
\usepackage{multirow}

\usepackage{enumitem}
\usepackage{mathtools}
\usepackage{steinmetz}
\usepackage{tikz}
\usepackage{circuitikz}
\usepackage{verbatim}
\usepackage{tfrupee}
\usepackage[breaklinks=true]{hyperref}

\usepackage{tkz-euclide}

\usetikzlibrary{calc,math}
\usepackage{listings}
    \usepackage{color}                                            %%
    \usepackage{array}                                            %%
    \usepackage{longtable}                                        %%
    \usepackage{calc}                                             %%
    \usepackage{multirow}                                         %%
    \usepackage{hhline}                                           %%
    \usepackage{ifthen}                                           %%
    \usepackage{lscape}     
\usepackage{multicol}
\usepackage{chngcntr}

\DeclareMathOperator*{\Res}{Res}

\renewcommand\thesection{\arabic{section}}
\renewcommand\thesubsection{\thesection.\arabic{subsection}}
\renewcommand\thesubsubsection{\thesubsection.\arabic{subsubsection}}

\renewcommand\thesectiondis{\arabic{section}}
\renewcommand\thesubsectiondis{\thesectiondis.\arabic{subsection}}
\renewcommand\thesubsubsectiondis{\thesubsectiondis.\arabic{subsubsection}}


\hyphenation{op-tical net-works semi-conduc-tor}
\def\inputGnumericTable{}                                 %%

\lstset{
%language=C,
frame=single, 
breaklines=true,
columns=fullflexible
}
\begin{document}


\newtheorem{theorem}{Theorem}[section]
\newtheorem{problem}{Problem}
\newtheorem{proposition}{Proposition}[section]
\newtheorem{lemma}{Lemma}[section]
\newtheorem{corollary}[theorem]{Corollary}
\newtheorem{example}{Example}[section]
\newtheorem{definition}[problem]{Definition}

\newcommand{\BEQA}{\begin{eqnarray}}
\newcommand{\EEQA}{\end{eqnarray}}
\newcommand{\define}{\stackrel{\triangle}{=}}
\bibliographystyle{IEEEtran}
\providecommand{\mbf}{\mathbf}
\providecommand{\pr}[1]{\ensuremath{\Pr\left(#1\right)}}
\providecommand{\qfunc}[1]{\ensuremath{Q\left(#1\right)}}
\providecommand{\sbrak}[1]{\ensuremath{{}\left[#1\right]}}
\providecommand{\lsbrak}[1]{\ensuremath{{}\left[#1\right.}}
\providecommand{\rsbrak}[1]{\ensuremath{{}\left.#1\right]}}
\providecommand{\brak}[1]{\ensuremath{\left(#1\right)}}
\providecommand{\lbrak}[1]{\ensuremath{\left(#1\right.}}
\providecommand{\rbrak}[1]{\ensuremath{\left.#1\right)}}
\providecommand{\cbrak}[1]{\ensuremath{\left\{#1\right\}}}
\providecommand{\lcbrak}[1]{\ensuremath{\left\{#1\right.}}
\providecommand{\rcbrak}[1]{\ensuremath{\left.#1\right\}}}
\theoremstyle{remark}
\newtheorem{rem}{Remark}
\newcommand{\sgn}{\mathop{\mathrm{sgn}}}
\providecommand{\abs}[1]{\left\vert#1\right\vert}
\providecommand{\res}[1]{\Res\displaylimits_{#1}} 
\providecommand{\norm}[1]{\left\lVert#1\right\rVert}
%\providecommand{\norm}[1]{\lVert#1\rVert}
\providecommand{\mtx}[1]{\mathbf{#1}}
\providecommand{\mean}[1]{E\left[ #1 \right]}
\providecommand{\fourier}{\overset{\mathcal{F}}{ \rightleftharpoons}}
%\providecommand{\hilbert}{\overset{\mathcal{H}}{ \rightleftharpoons}}
\providecommand{\system}{\overset{\mathcal{H}}{ \longleftrightarrow}}
	%\newcommand{\solution}[2]{\textbf{Solution:}{#1}}
\newcommand{\solution}{\noindent \textbf{Solution: }}
\newcommand{\cosec}{\,\text{cosec}\,}
\providecommand{\dec}[2]{\ensuremath{\overset{#1}{\underset{#2}{\gtrless}}}}
\newcommand{\myvec}[1]{\ensuremath{\begin{pmatrix}#1\end{pmatrix}}}
\newcommand{\mydet}[1]{\ensuremath{\begin{vmatrix}#1\end{vmatrix}}}
\numberwithin{equation}{subsection}
\makeatletter
\@addtoreset{figure}{problem}
\makeatother
\let\StandardTheFigure\thefigure
\let\vec\mathbf
\renewcommand{\thefigure}{\theproblem}
\def\putbox#1#2#3{\makebox[0in][l]{\makebox[#1][l]{}\raisebox{\baselineskip}[0in][0in]{\raisebox{#2}[0in][0in]{#3}}}}
     \def\rightbox#1{\makebox[0in][r]{#1}}
     \def\centbox#1{\makebox[0in]{#1}}
     \def\topbox#1{\raisebox{-\baselineskip}[0in][0in]{#1}}
     \def\midbox#1{\raisebox{-0.5\baselineskip}[0in][0in]{#1}}
\vspace{3cm}
\title{Assignment 3}
\author{Adarsh Srivastava}
\maketitle
\newpage
\bigskip
%\renewcommand{\thefigure}{\theenumi}
\renewcommand{\thetable}{\theenumi}
The link to the solution is
\begin{lstlisting}
 https://github.com/Adarsh1310/EE5609
\end{lstlisting}
\begin{abstract}
This documents depicts that matrix multiplication is non commutative.
\end{abstract}
\section{\textbf{Problem}}
Show That\\
\\
\myvec{1 & 2 & 3\\ 0 &1&0\\1&1&0} \myvec{-1 & 1 & 0\\ -1 &1&0\\2&3&4}\not=\myvec{-1 & 1 & 0\\ -1 &1&0\\2&3&4}\myvec{1 & 2 & 3\\ 0 &1&0\\1&1&0}

\section{\textbf{Explanation}}
\begin{description}
\item[$\cdot$ ]
If Matrices $\bf{M}$ and $\bf{N}$ are rectangular then we can directly say that multiplication of these matrices can't be commutative.\\
\item[$\cdot$]
If Matrices $\bf{M}$ and $\bf{N}$ are square matrices then they are generally not commutative because:-\\

1. While computing $\bf{MN}$, row element of $\bf{M}$ will be multiplied with columns element of $\bf{N}$ to find the elements of resultant matrix.\\
\\
2. While computing $\bf{NM}$, rows elements of $\bf{N}$ will be multiplied with column elements of $\bf{M}$ and then summed to find the elements of resultant matrix.\\
\end{description}
Conditions in which matrix multiplication becomes commutative are as follow:
\\



1.Multiplication of $\vec{M}$ and $\vec{N}$ is commutative if both $\vec{M}$ and $\vec{N}$ are simultaneously diagonalisable.A matrix is said to be diagonalisable if there exist a matrix $\vec{C}$ and a matrix $\vec{D}$ such that

$\vec{M}$
=
$\bf{C^{-1}$
$\vec{M}$
$\vec{C}$.

\\
\\2. Multiplication of Matrix is commutative when one or both of the matrices are $\vec{I}$
\\
\\3.If one matrix is a scalar multiple of others then also the multiplication will be commutative.
\\
4.If one of the matrix can be derived by some power of the other matrix then the multiplication of these two matrices will be commutative.

\section{\textbf{Solution}}

Let's name the ,matrices as:-\\ 
\\
$\vec{M}$ = \myvec{1 & 2 & 3\\ 0 &1&0\\1&1&0} and $\vec{N}$ = \myvec{-1 & 1 & 0\\ -1 &1&0\\2&3&4}\\
\\ To prove that multiplication is non commutative we have to show that
\begin{align}
\vec{M}\vec{N} \not= \vec{N}\vec{M}
\end{align}

\\
\bftext{Solving L.H.S of Equation 3.0.1}
\\
\begin{align}
\vec{M}\vec{N} &= \myvec{1 & 2 & 3\\ 0 &1&0\\1&1&0} \myvec{-1 & 1 & 0\\ -1 &1&0\\2&3&4}
\end{align}
\\
{\tiny
\begin{align}
=\myvec{1\times-1+ 2\times-1+ 3\times2 &1\times1+ 2\times1+ 3\times3 & 1\times0+ 2\times0+ 3\times4\\
0\times-1+ 1\times-1+ 0\times2 &0\times1+ 1\times1+ 0\times3 & 0\times0+ 1\times0+ 0\times4\\
1\times-1+ 1\times-1+ 0\times2 &1\times1+ 1\times1+ 0\times3 & 1\times0+ 1\times0+ 0\times4\\}
\end{align*}
}
\begin{align}
\vec{M}\vec{N} &= \myvec{3 & 12 & 12\\-1 & 1 &0\\-2&2&0}\
\end{align}
\bftext{Solving R.H.S of Equation 3.0.1}
\\
\begin{align}
\vec{N}\vec{M} &=  \myvec{-1 & 1 & 0\\ -1 &1&0\\2&3&4}
\myvec{1 & 2 & 3\\ 0 &1&0\\1&1&0} 
\end{align}
\\
{\tiny
\begin{align}
=\myvec{-1\times1+ 1\times0+ 0\times1 &-1\times2+ 1\times1+ 0\times1 & -1\times3+ 1\times0+ 0\times0\\
-1\times1+ 1\times0+ 0\times1 &-1\times2+ 1\times1+ 0\times1 & -1\times3+ 1\times0+ 0\times0\\
2\times1+ 3\times0+ 4\times1 &2\times2+ 3\times1+ 4\times1 & 2\times3+ 3\times0+ 4\times0\\\ }
\end{align*}
}
\begin{align}
\vec{N}\vec{M} &= \myvec{-1 & -1 & -3\\-1 & -1&-3\\6&11&6}\
\end{align}
From Equations 3.0.4 and 3.0.7 we can clearly see that R.H.S $\not=$ L.H.S and Hence, Matrix multiplication is non commutative

\end{document}
