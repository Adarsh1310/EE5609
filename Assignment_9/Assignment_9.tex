\documentclass[journal,12pt,twocolumn]{IEEEtran}
 \usepackage{setspace}
 \usepackage{gensymb}
 \usepackage{graphicx}
 \singlespacing
\graphicspath{ {/user/adarshsrivastava/desktop/Matrix Theory/Assignemnt_7} }
 \usepackage[cmex10]{amsmath}
 \usepackage{amsthm}
 \usepackage{hyperref}
 \usepackage{mathrsfs}
 \usepackage{txfonts}
 \usepackage{stfloats}
 \usepackage{bm}
 \usepackage{cite}
 \usepackage{cases}
 \usepackage{subfig}
 \usepackage{longtable}
 \usepackage{multirow}
 \usepackage{enumitem}
 \usepackage{mathtools}
 \usepackage{steinmetz}
 \usepackage{tikz}
 \usepackage{circuitikz}
 \usepackage{verbatim}
 \usepackage{tfrupee}
 \usepackage[breaklinks=true]{hyperref}
 \usepackage{tkz-euclide}
 \usetikzlibrary{calc,math}
 \usepackage{listings}
     \usepackage{color}                                            %%
     \usepackage{array}                                            %%
     \usepackage{longtable}                                        %%
     \usepackage{calc}                                             %%
     \usepackage{multirow}                                         %%
     \usepackage{hhline}                                           %%
     \usepackage{ifthen}                                           %%
     \usepackage{lscape}     
 \usepackage{multicol}
 \usepackage{chngcntr}
 \DeclareMathOperator*{\Res}{Res}
 \renewcommand\thesection{\arabic{section}}
 \renewcommand\thesubsection{\thesection.\arabic{subsection}}
 \renewcommand\thesubsubsection{\thesubsection.\arabic{subsubsection}}
 \renewcommand\thesectiondis{\arabic{section}}
 \renewcommand\thesubsectiondis{\thesectiondis.\arabic{subsection}}
 \renewcommand\thesubsubsectiondis{\thesubsectiondis.\arabic{subsubsection}}
 \hyphenation{op-tical net-works semi-conduc-tor}
 \def\inputGnumericTable{}                                 %%
 \lstset{
 %language=C,
 frame=single, 
 breaklines=true,
 columns=fullflexible
 }
 \begin{document}
 \newtheorem{theorem}{Theorem}[section]
 \newtheorem{problem}{Problem}
 \newtheorem{proposition}{Proposition}[section]
 \newtheorem{lemma}{Lemma}[section]
 \newtheorem{corollary}[theorem]{Corollary}
 \newtheorem{example}{Example}[section]
 \newtheorem{definition}[problem]{Definition}
 \newcommand{\BEQA}{\begin{eqnarray}}
 \newcommand{\EEQA}{\end{eqnarray}}
 \newcommand{\define}{\stackrel{\triangle}{=}}
 \bibliographystyle{IEEEtran}
 \providecommand{\mbf}{\mathbf}
 \providecommand{\pr}[1]{\ensuremath{\Pr\left(#1\right)}}
 \providecommand{\qfunc}[1]{\ensuremath{Q\left(#1\right)}}
 \providecommand{\sbrak}[1]{\ensuremath{{}\left[#1\right]}}
 \providecommand{\lsbrak}[1]{\ensuremath{{}\left[#1\right.}}
 \providecommand{\rsbrak}[1]{\ensuremath{{}\left.#1\right]}}
 \providecommand{\brak}[1]{\ensuremath{\left(#1\right)}}
 \providecommand{\lbrak}[1]{\ensuremath{\left(#1\right.}}
 \providecommand{\rbrak}[1]{\ensuremath{\left.#1\right)}}
 \providecommand{\cbrak}[1]{\ensuremath{\left\{#1\right\}}}
 \providecommand{\lcbrak}[1]{\ensuremath{\left\{#1\right.}}
 \providecommand{\rcbrak}[1]{\ensuremath{\left.#1\right\}}}
 \theoremstyle{remark}
 \newtheorem{rem}{Remark}
 \newcommand{\sgn}{\mathop{\mathrm{sgn}}}
 \providecommand{\abs}[1]{\left\vert#1\right\vert}
 \providecommand{\res}[1]{\Res\displaylimits_{#1}} 
 \providecommand{\norm}[1]{\left\lVert#1\right\rVert}
 %\providecommand{\norm}[1]{\lVert#1\rVert}
 \providecommand{\mtx}[1]{\mathbf{#1}}
 \providecommand{\mean}[1]{E\left[ #1 \right]}
 \providecommand{\fourier}{\overset{\mathcal{F}}{ \rightleftharpoons}}
 %\providecommand{\hilbert}{\overset{\mathcal{H}}{ \rightleftharpoons}}
 \providecommand{\system}{\overset{\mathcal{H}}{ \longleftrightarrow}}
 	%\newcommand{\solution}[2]{\textbf{Solution:}{#1}}
 \newcommand{\solution}{\noindent \textbf{Solution: }}
 \newcommand{\cosec}{\,\text{cosec}\,}
 \providecommand{\dec}[2]{\ensuremath{\overset{#1}{\underset{#2}{\gtrless}}}}
 \newcommand{\myvec}[1]{\ensuremath{\begin{pmatrix}#1\end{pmatrix}}}
 \newcommand{\mydet}[1]{\ensuremath{\begin{vmatrix}#1\end{vmatrix}}}
 \numberwithin{equation}{subsection}
 \makeatletter
 \@addtoreset{figure}{problem}
 \makeatother
 \let\StandardTheFigure\thefigure
 \let\vec\mathbf
 \renewcommand{\thefigure}{\theproblem}
 \def\putbox#1#2#3{\makebox[0in][l]{\makebox[#1][l]{}\raisebox{\baselineskip}[0in][0in]{\raisebox{#2}[0in][0in]{#3}}}}
      \def\rightbox#1{\makebox[0in][r]{#1}}
      \def\centbox#1{\makebox[0in]{#1}}
      \def\topbox#1{\raisebox{-\baselineskip}[0in][0in]{#1}}
      \def\midbox#1{\raisebox{-0.5\baselineskip}[0in][0in]{#1}}
 \vspace{3cm}
 \title{Assignment 9}
 \author{Adarsh Srivastava}
 \maketitle
 \newpage
 \bigskip
 %\renewcommand{\thefigure}{\theenumi}
 \renewcommand{\thetable}{\theenumi}
 The link to the solution is
 \begin{lstlisting}
  https://github.com/Adarsh1310/EE5609
 \end{lstlisting}
 \begin{abstract}
 This documents solves a Singular Value decomposition problem.
 \end{abstract}
  \section{\textbf{Problem}}
Find the shortest distance between the lines
\begin{align}
\vec{x}=\myvec{1\\1\\0}+\lambda_1\myvec{2\\-1\\1}\\
\vec{x}=\myvec{2\\1\\-1}+\lambda_2\myvec{3\\-5\\2}
\end{align}
 \section{\textbf{Solution}}
The lines will intersect if
\begin{align}
\myvec{1\\1\\0}+\lambda_1\myvec{2\\-1\\1}=\myvec{2\\1\\-1}+\lambda_2\myvec{3\\-5\\2}\\
\myvec{2&3\\-1&-5\\1&2}\myvec{x_1\\x_2}=\myvec{1\\0\\-1}\\
\vec{M}\vec{x}=\vec{b}
\end{align}
Since the rank of augmented matrix will be 3. We can say that lines do not intersect.
\begin{align}
\vec{M}=\vec{U}\vec{S}\vec{V}^T\label{2.0.4}
\end{align}
Where the columns of $\vec{V}$ are the eigenvectors of $\vec{A}^T\vec{A}$ ,the columns of $\vec{U}$ are the eigenvectors of $\vec{A}\vec{A}^T$ and $\vec{S}$ is diagonal matrix of singular value of eigenvalues of $\vec{A}^T\vec{A}$.
\begin{align}
\vec{M}^T\vec{M}=\myvec{6&13\\13&38}\\
\vec{M}\vec{M}^T=\myvec{13&-17&8\\-17&26&-11\\8&-11&5}
\end{align}
Calculating $\vec{M}$^T$\vec{M}.
\begin{align}
\begin{vmatrix}
6-\lambda&13\\13&38-\lambda
\end{vmatrix}
\lambda^2-44\lambda+59=0\\
\lambda_1=-5\sqrt{17}+22,\lambda_2=5\sqrt{17}+22
\end{align}
Eigen vectors of $\vec{M}$$\vec{M}^T.$
\begin{align}
\begin{vmatrix}
13-\lambda&-17&8\\17&26-\lambda&-11\\8&-11&5-\lambda
\end{vmatrix}
-\lambda^3+44\lambda^2-59\lambda=0\\
\lambda_4=-5\sqrt{17}+22,\lambda_3=5\sqrt{17}+22,\lambda_5=0,
\end{align}
Hence,The eigenvectors will be
\begin{align}
\vec{u}_2=\myvec{\frac{\sqrt{17}+12}{5}\\\frac{3\sqrt{17}+1}{5}\\1}
\vec{u}_1=\myvec{\frac{-\sqrt{17}+12}{5}\\\frac{-3\sqrt{17}+1}{5}\\1}
\vec{u}_3=\myvec{\frac{-3}{7}\\\frac{1}{7}\\1}
\end{align}
Normalising the eigenvectors
\begin{align}
l_1=\sqrt{{\left(\frac{12-\sqrt{17}}{5}\right)}^2+{\left(\frac{1-3\sqrt{17}}{5}\right)}^2+1^2}\\
\vec{u}_1=\myvec{\frac{-\sqrt{17}+12}{\sqrt{340-20\sqrt{17}}}\\\frac{-3\sqrt{17}+1}{\sqrt{340-20\sqrt{17}}}\\\frac{5}{\sqrt{340-20\sqrt{17}}}}\\
\end{align}
\begin{align}
l_2=\sqrt{\left({\frac{\sqrt{17}+12}{5}}\right)^2+\left({\frac{3\sqrt{17}+1}{5}}\right)^2+1^2}\\
\vec{u}_2=\frac{5}{\sqrt{340+20\sqrt{7}}}\myvec{\frac{\sqrt{17}+12}{5}\\\frac{3\sqrt{17}+1}{5}\\1}\\
\vec{u}_2=\myvec{\frac{\sqrt{17}+12}{\sqrt{340+20\sqrt{17}}}\\\frac{3\sqrt{17}+1}{\sqrt{340+20\sqrt{17}}}\\\frac{5}{\sqrt{340+20\sqrt{17}}}}
\end{align}
\begin{align}
l_3=\sqrt{\left({\frac{-3}{7}}\right)^2+\left({\frac{1}{7}}\right)^2+1^2}\\
\vec{u}_3=\frac{7}{\sqrt{59}}\myvec{\frac{-3}{7}\\\frac{1}{7}}\\
\vec{u}_3=\myvec{\frac{-3}{\sqrt{59}}\\\frac{1}{\sqrt{59}}\\\frac{7}{\sqrt{59}}\\\frac{7}{\sqrt{59}}}
\end{align}
\begin{align}
\vec{U}=\myvec{\frac{-\sqrt{17}+12}{\sqrt{340-20\sqrt{17}}}&\frac{\sqrt{17}+12}{\sqrt{340+20\sqrt{17}}}&\frac{-3}{\sqrt{59}}\\\frac{-3\sqrt{17}+1}{\sqrt{340-20\sqrt{17}}}&\frac{3\sqrt{17}+1}{\sqrt{340+20\sqrt{17}}}&\frac{1}{\sqrt{59}}\\\frac{7}{\sqrt{59}}\\\frac{5}{\sqrt{340-20\sqrt{17}}}&\frac{5}{\sqrt{340+20\sqrt{17}}}&\frac{7}{\sqrt{59}}}
\end{align}
Now,
\begin{align}
\vec{S}=\myvec{\sqrt{5\sqrt{17}+22}&0\\0&\sqrt{-5\sqrt{17}+22}\\0&0}
\end{align}
Now, \vec{V}=$\vec{M}^T$$\frac{\vec{u_i}}{\sqrt{\lambda_i}}$
\begin{align}
\vec{V}=\myvec{\frac{\sqrt{17}+28}{\sqrt{340-20\sqrt{17}}\sqrt{5\sqrt{17}+22}}&\frac{-\sqrt{17}+26}{\sqrt{340+20\sqrt{17}}\sqrt{-5\sqrt{17}+22}}\\\frac{-\sqrt{17}+26}{\sqrt{340+20\sqrt{17}}\sqrt{-5\sqrt{17}+22}}&\frac{-\sqrt{17}-28}{\sqrt{340-20\sqrt{17}}\sqrt{5\sqrt{17}+22}}}
\end{align}
So,from equation \eqref{2.0.4}
\begin{align}
\begin{multlined}
\myvec{2&3\\-1&-5\\1&2}=\\
\myvec{\frac{-\sqrt{17}+12}{\sqrt{340-20\sqrt{17}}}&\frac{\sqrt{17}+12}{\sqrt{340+20\sqrt{17}}}&\frac{-3}{\sqrt{59}}\\\frac{-3\sqrt{17}+1}{\sqrt{340-20\sqrt{17}}}&\frac{3\sqrt{17}+1}{\sqrt{340+20\sqrt{17}}}&\frac{1}{\sqrt{59}}\\\frac{5}{\sqrt{340-20\sqrt{17}}}&\frac{5}{\sqrt{340+20\sqrt{17}}}&\frac{7}{\sqrt{59}}}\\\myvec{\sqrt{5\sqrt{17}+22}&0\\0&\sqrt{-5\sqrt{17}+22}\\0&0}\\\myvec{\frac{\sqrt{17}+28}{\sqrt{340-20\sqrt{17}}\sqrt{5\sqrt{17}+22}}&\frac{-\sqrt{17}+26}{\sqrt{340+20\sqrt{17}}\sqrt{-5\sqrt{17}+22}}\\\frac{-\sqrt{17}+26}{\sqrt{340+20\sqrt{17}}\sqrt{-5\sqrt{17}+22}}&\frac{-\sqrt{17}-28}{\sqrt{340-20\sqrt{17}}\sqrt{5\sqrt{17}+22}}}^T
\end{multlined}
\end{align}
Now, Finding Moore-Penrose Pseudo inverse of \vec{S}
\begin{align}
\vec{S}_+=\myvec{\frac{1}{\sqrt{5\sqrt{17}+22}}&0&0\\0&\frac{1}{\sqrt{-5\sqrt{17}+22}}&0}
\end{align}
We,know that,
\vec{x}=$\vec{V}(\vec{S}_+(\vec{U}^T\vec{b}))$
\begin{align}
\vec{U}^T\vec{b}=\myvec{\frac{-\sqrt{17}+7}{\sqrt{340-20\sqrt{17}}}\\\frac{\sqrt{17}+7}{\sqrt{340+20\sqrt{17}}}\\\frac{-10}{\sqrt{59}}}\\
\vec{S}_+(\vec{U}^T\vec{b})=\myvec{\frac{-\sqrt{17}+7}{\sqrt{340-20\sqrt{17}}\sqrt{5\sqrt{17}+22}}\\\frac{\sqrt{17}+7}{\sqrt{340+20\sqrt{17}}\sqrt{-5\sqrt{17}+22}}}\\
\begin{multlined}
\vec{x}=\myvec{\frac{\sqrt{17}+28}{\sqrt{340-20\sqrt{17}}\sqrt{5\sqrt{17}+22}}&\frac{-\sqrt{17}+26}{\sqrt{340+20\sqrt{17}}\sqrt{-5\sqrt{17}+22}}\\\frac{-\sqrt{17}+26}{\sqrt{340+20\sqrt{17}}\sqrt{-5\sqrt{17}+22}}&\frac{-\sqrt{17}-28}{\sqrt{340-20\sqrt{17}}\sqrt{5\sqrt{17}+22}}}\\
\myvec{\frac{-\sqrt{17}+7}{\sqrt{340-20\sqrt{17}}\sqrt{5\sqrt{17}+22}}\\\frac{\sqrt{17}+7}{\sqrt{340+20\sqrt{17}}\sqrt{-5\sqrt{17}+22}}}
\end{multlined}\\
\vec{x}=\myvec{\frac{971\sqrt{17}+127323}{320960}\\\frac{-68\sqrt{17}-14}{\sqrt{108800}\sqrt{59}}}
\end{align}
Now, Verifying the values using
\begin{align}
\vec{M}^T\vec{M}\vec{x} = \vec{M}^T\vec{b}
\end{align}
Taking R.H.S
\begin{align}
=\myvec{1\\1}
\end{align}
Taking L.H.S
\begin{align}
=\myvec{6&13\\13&38}\myvec{\frac{1942\sqrt{17}+254646}{64,19,20}\\\frac{-68\sqrt{17}-14}{\sqrt{108800}\sqrt{59}}}\\
=\myvec{\frac{5826\sqrt{17}+763938}{320960}+\frac{-884\sqrt{17}-182}{\sqrt{108800}\sqrt{59}}\\\frac{12623\sqrt{17}+1655199}{320960}+\frac{-2584\sqrt{17}-5322}{\sqrt{108800}\sqrt{59}}}\\
=\myvec{1\\1}
\end{align}
 \end{document}