\documentclass[journal,12pt,twocolumn]{IEEEtran}
  \usepackage{setspace}
  \usepackage{gensymb}
  \usepackage{graphicx}
  \singlespacing

  \usepackage[cmex10]{amsmath}
  \usepackage{amsthm}
  \usepackage{hyperref}
  \usepackage{mathrsfs}
  \usepackage{txfonts}
  \usepackage{stfloats}
  \usepackage{bm}
  \usepackage{cite}
  \usepackage{cases}
  \usepackage{subfig}
  \usepackage{longtable}
  \usepackage{multirow}
  \usepackage{enumitem}
  \usepackage{mathtools}
  \usepackage{steinmetz}
  \usepackage{tikz}
  \usepackage{circuitikz}
  \usepackage{verbatim}
  \usepackage{tfrupee}
  \usepackage[breaklinks=true]{hyperref}
  \usepackage{tkz-euclide}
  \usetikzlibrary{calc,math}
  \usepackage{listings}
      \usepackage{color}                                            %%
      \usepackage{array}                                            %%
      \usepackage{longtable}                                        %%
      \usepackage{calc}                                             %%
      \usepackage{multirow}                                         %%
      \usepackage{hhline}                                           %%
      \usepackage{ifthen}                                           %%
      \usepackage{lscape}     
  \usepackage{multicol}
  \usepackage{chngcntr}
  \DeclareMathOperator*{\Res}{Res}
  \renewcommand\thesection{\arabic{section}}
  \renewcommand\thesubsection{\thesection.\arabic{subsection}}
  \renewcommand\thesubsubsection{\thesubsection.\arabic{subsubsection}}
  \renewcommand\thesectiondis{\arabic{section}}
  \renewcommand\thesubsectiondis{\thesectiondis.\arabic{subsection}}
  \renewcommand\thesubsubsectiondis{\thesubsectiondis.\arabic{subsubsection}}
  \hyphenation{op-tical net-works semi-conduc-tor}
  \def\inputGnumericTable{}                                 %%
  \lstset{
  %language=C,
  frame=single, 
  breaklines=true,
  columns=fullflexible
  }
  \begin{document}
  \newtheorem{theorem}{Theorem}[section]
  \newtheorem{problem}{Problem}
  \newtheorem{proposition}{Proposition}[section]
  \newtheorem{lemma}{Lemma}[section]
  \newtheorem{corollary}[theorem]{Corollary}
  \newtheorem{example}{Example}[section]
  \newtheorem{definition}[problem]{Definition}
  \newcommand{\BEQA}{\begin{eqnarray}}
  \newcommand{\EEQA}{\end{eqnarray}}
  \newcommand{\define}{\stackrel{\triangle}{=}}
  \bibliographystyle{IEEEtran}
  \providecommand{\mbf}{\mathbf}
  \providecommand{\pr}[1]{\ensuremath{\Pr\left(#1\right)}}
  \providecommand{\qfunc}[1]{\ensuremath{Q\left(#1\right)}}
  \providecommand{\sbrak}[1]{\ensuremath{{}\left[#1\right]}}
  \providecommand{\lsbrak}[1]{\ensuremath{{}\left[#1\right.}}
  \providecommand{\rsbrak}[1]{\ensuremath{{}\left.#1\right]}}
  \providecommand{\brak}[1]{\ensuremath{\left(#1\right)}}
  \providecommand{\lbrak}[1]{\ensuremath{\left(#1\right.}}
  \providecommand{\rbrak}[1]{\ensuremath{\left.#1\right)}}
  \providecommand{\cbrak}[1]{\ensuremath{\left\{#1\right\}}}
  \providecommand{\lcbrak}[1]{\ensuremath{\left\{#1\right.}}
  \providecommand{\rcbrak}[1]{\ensuremath{\left.#1\right\}}}
  \theoremstyle{remark}
  \newtheorem{rem}{Remark}
  \newcommand{\sgn}{\mathop{\mathrm{sgn}}}
  \providecommand{\abs}[1]{\left\vert#1\right\vert}
  \providecommand{\res}[1]{\Res\displaylimits_{#1}} 
  \providecommand{\norm}[1]{\left\lVert#1\right\rVert}
  %\providecommand{\norm}[1]{\lVert#1\rVert}
  \providecommand{\mtx}[1]{\mathbf{#1}}
  \providecommand{\mean}[1]{E\left[ #1 \right]}
  \providecommand{\fourier}{\overset{\mathcal{F}}{ \rightleftharpoons}}
  %\providecommand{\hilbert}{\overset{\mathcal{H}}{ \rightleftharpoons}}
  \providecommand{\system}{\overset{\mathcal{H}}{ \longleftrightarrow}}
  	%\newcommand{\solution}[2]{\textbf{Solution:}{#1}}
  \newcommand{\solution}{\noindent \textbf{Solution: }}
  \newcommand{\cosec}{\,\text{cosec}\,}
  \providecommand{\dec}[2]{\ensuremath{\overset{#1}{\underset{#2}{\gtrless}}}}
  \newcommand{\myvec}[1]{\ensuremath{\begin{pmatrix}#1\end{pmatrix}}}
  \newcommand{\mydet}[1]{\ensuremath{\begin{vmatrix}#1\end{vmatrix}}}
  \numberwithin{equation}{subsection}
  \makeatletter
  \@addtoreset{figure}{problem}
  \makeatother
  \let\StandardTheFigure\thefigure
  \let\vec\mathbf
  \renewcommand{\thefigure}{\theproblem}
  \def\putbox#1#2#3{\makebox[0in][l]{\makebox[#1][l]{}\raisebox{\baselineskip}[0in][0in]{\raisebox{#2}[0in][0in]{#3}}}}
       \def\rightbox#1{\makebox[0in][r]{#1}}
       \def\centbox#1{\makebox[0in]{#1}}
       \def\topbox#1{\raisebox{-\baselineskip}[0in][0in]{#1}}
       \def\midbox#1{\raisebox{-0.5\baselineskip}[0in][0in]{#1}}
  \vspace{3cm}
  \title{Assignment 22}
  \author{Adarsh Srivastava}
  \maketitle
  \newpage
  \bigskip
  %\renewcommand{\thefigure}{\theenumi}
  \renewcommand{\thetable}{\theenumi}
  The link to the solution is
  \begin{lstlisting}
   https://github.com/Adarsh1310/EE5609
  \end{lstlisting}
  \begin{abstract}
  This documents solves a problem based on Lagrange Interpolation
  \end{abstract}
\section{\textbf{Problem}}
Let $\mathbb{F}$ be the field of real numbers,
\begin{align*}
\vec{A}=\myvec{2&0&0&0\\0&2&0&0\\0&0&3&0\\0&0&0&1}\\
p=(x-2)(x-3)(x-1)
\end{align*}
\begin{enumerate}
\item{Show that $p(\vec{A})=0$}
\item{Let $P_1,P_2,P_3$ be the Lagrange polynomials for $t_1=2, t_2=3,t_3=1$. Compute $E_i=P_i(\vec{A})$, i=1,2,3}
\end{enumerate}
\section{\textbf{Solution}}
\begin{enumerate}
\item{
We have been given that,\begin{align}p(x)=(x-2)(x-3)(x-1)\end{align} Now to find p($\vec{A}$),
\begin{align}
p(\vec{A})&=(\vec{A}-2)(\vec{A}-3)(\vec{A}-1)\end{align}
\begin{multline}
&=\myvec{0&0&0&0\\0&0&0&0\\0&0&1&0\\0&0&0&-1}\myvec{-1&0&0&0\\0&-1&0&0\\0&0&0&0\\0&0&0&-2}\times\\\myvec{1&0&0&0\\0&1&0&0\\0&0&2&0\\0&0&0&0}
\end{multline}
\begin{align}
&=\myvec{0&0&0&0\\0&0&0&0\\0&0&1&0\\0&0&0&-1}\myvec{-1&0&0&0\\0&-1&0&0\\0&0&0&0\\0&0&0&0}\\
&=\myvec{0&0&0&0\\0&0&0&0\\0&0&0&0\\0&0&0&0}
\end{align}
Hence we proved that p($\vec{A}$)=0 }
\item{Using Lagrange Interpolation,
\begin{align}
P_1(x)=\frac{(x-3)(x-1)}{(2-3)(2-1)}\\
=-(x-3)(x-1)\\
P_2(x)=\frac{(x-2)(x-1)}{(3-2)(3-1)}\\
=\frac{(x-2)(x-1)}{2}\\
P_3(x)=\frac{(x-2)(x-3)}{(1-2)(1-3)}\\
=\frac{(x-2)(x-3)}{2}
\end{align}
Now,Substituting the value of $\vec{A}$,
\begin{align}
P_1(\vec{A})=-\myvec{-1&0&0&0\\0&-1&0&0\\0&0&0&0\\0&0&0&-2}\myvec{1&0&0&0\\0&1&0&0\\0&0&2&0\\0&0&0&0}\\=\myvec{1&0&0&0\\0&1&0&0\\0&0&0&0\\0&0&0&0}\\
P_2(\vec{A})=\frac{1}{2}\myvec{0&0&0&0\\0&0&0&0\\0&0&1&0\\0&0&0&-1}\myvec{1&0&0&0\\0&1&0&0\\0&0&2&0\\0&0&0&0}\\=\myvec{0&0&0&0\\0&0&0&0\\0&0&1&0\\0&0&0&0}\\
P_3(\vec{A})=\frac{1}{2}\myvec{0&0&0&0\\0&0&0&0\\0&0&1&0\\0&0&0&-1}\myvec{-1&0&0&0\\0&-1&0&0\\0&0&0&0\\0&0&0&-2}\\=\myvec{0&0&0&0\\0&0&0&0\\0&0&0&0\\0&0&0&1}
\end{align}}
\end{enumerate}
\end{document}

