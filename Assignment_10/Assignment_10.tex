\documentclass[journal,12pt,twocolumn]{IEEEtran}
 \usepackage{setspace}
 \usepackage{gensymb}
 \usepackage{graphicx}
 \singlespacing
\graphicspath{ {/user/adarshsrivastava/desktop/Matrix Theory/Assignemnt_7} }
 \usepackage[cmex10]{amsmath}
 \usepackage{amsthm}
 \usepackage{hyperref}
 \usepackage{mathrsfs}
 \usepackage{txfonts}
 \usepackage{stfloats}
 \usepackage{bm}
 \usepackage{cite}
 \usepackage{cases}
 \usepackage{subfig}
 \usepackage{longtable}
 \usepackage{multirow}
 \usepackage{enumitem}
 \usepackage{mathtools}
 \usepackage{steinmetz}
 \usepackage{tikz}
 \usepackage{circuitikz}
 \usepackage{verbatim}
 \usepackage{tfrupee}
 \usepackage[breaklinks=true]{hyperref}
 \usepackage{tkz-euclide}
 \usetikzlibrary{calc,math}
 \usepackage{listings}
     \usepackage{color}                                            %%
     \usepackage{array}                                            %%
     \usepackage{longtable}                                        %%
     \usepackage{calc}                                             %%
     \usepackage{multirow}                                         %%
     \usepackage{hhline}                                           %%
     \usepackage{ifthen}                                           %%
     \usepackage{lscape}     
 \usepackage{multicol}
 \usepackage{chngcntr}
 \DeclareMathOperator*{\Res}{Res}
 \renewcommand\thesection{\arabic{section}}
 \renewcommand\thesubsection{\thesection.\arabic{subsection}}
 \renewcommand\thesubsubsection{\thesubsection.\arabic{subsubsection}}
 \renewcommand\thesectiondis{\arabic{section}}
 \renewcommand\thesubsectiondis{\thesectiondis.\arabic{subsection}}
 \renewcommand\thesubsubsectiondis{\thesubsectiondis.\arabic{subsubsection}}
 \hyphenation{op-tical net-works semi-conduc-tor}
 \def\inputGnumericTable{}                                 %%
 \lstset{
 %language=C,
 frame=single, 
 breaklines=true,
 columns=fullflexible
 }
 \begin{document}
 \newtheorem{theorem}{Theorem}[section]
 \newtheorem{problem}{Problem}
 \newtheorem{proposition}{Proposition}[section]
 \newtheorem{lemma}{Lemma}[section]
 \newtheorem{corollary}[theorem]{Corollary}
 \newtheorem{example}{Example}[section]
 \newtheorem{definition}[problem]{Definition}
 \newcommand{\BEQA}{\begin{eqnarray}}
 \newcommand{\EEQA}{\end{eqnarray}}
 \newcommand{\define}{\stackrel{\triangle}{=}}
 \bibliographystyle{IEEEtran}
 \providecommand{\mbf}{\mathbf}
 \providecommand{\pr}[1]{\ensuremath{\Pr\left(#1\right)}}
 \providecommand{\qfunc}[1]{\ensuremath{Q\left(#1\right)}}
 \providecommand{\sbrak}[1]{\ensuremath{{}\left[#1\right]}}
 \providecommand{\lsbrak}[1]{\ensuremath{{}\left[#1\right.}}
 \providecommand{\rsbrak}[1]{\ensuremath{{}\left.#1\right]}}
 \providecommand{\brak}[1]{\ensuremath{\left(#1\right)}}
 \providecommand{\lbrak}[1]{\ensuremath{\left(#1\right.}}
 \providecommand{\rbrak}[1]{\ensuremath{\left.#1\right)}}
 \providecommand{\cbrak}[1]{\ensuremath{\left\{#1\right\}}}
 \providecommand{\lcbrak}[1]{\ensuremath{\left\{#1\right.}}
 \providecommand{\rcbrak}[1]{\ensuremath{\left.#1\right\}}}
 \theoremstyle{remark}
 \newtheorem{rem}{Remark}
 \newcommand{\sgn}{\mathop{\mathrm{sgn}}}
 \providecommand{\abs}[1]{\left\vert#1\right\vert}
 \providecommand{\res}[1]{\Res\displaylimits_{#1}} 
 \providecommand{\norm}[1]{\left\lVert#1\right\rVert}
 %\providecommand{\norm}[1]{\lVert#1\rVert}
 \providecommand{\mtx}[1]{\mathbf{#1}}
 \providecommand{\mean}[1]{E\left[ #1 \right]}
 \providecommand{\fourier}{\overset{\mathcal{F}}{ \rightleftharpoons}}
 %\providecommand{\hilbert}{\overset{\mathcal{H}}{ \rightleftharpoons}}
 \providecommand{\system}{\overset{\mathcal{H}}{ \longleftrightarrow}}
 	%\newcommand{\solution}[2]{\textbf{Solution:}{#1}}
 \newcommand{\solution}{\noindent \textbf{Solution: }}
 \newcommand{\cosec}{\,\text{cosec}\,}
 \providecommand{\dec}[2]{\ensuremath{\overset{#1}{\underset{#2}{\gtrless}}}}
 \newcommand{\myvec}[1]{\ensuremath{\begin{pmatrix}#1\end{pmatrix}}}
 \newcommand{\mydet}[1]{\ensuremath{\begin{vmatrix}#1\end{vmatrix}}}
 \numberwithin{equation}{subsection}
 \makeatletter
 \@addtoreset{figure}{problem}
 \makeatother
 \let\StandardTheFigure\thefigure
 \let\vec\mathbf
 \renewcommand{\thefigure}{\theproblem}
 \def\putbox#1#2#3{\makebox[0in][l]{\makebox[#1][l]{}\raisebox{\baselineskip}[0in][0in]{\raisebox{#2}[0in][0in]{#3}}}}
      \def\rightbox#1{\makebox[0in][r]{#1}}
      \def\centbox#1{\makebox[0in]{#1}}
      \def\topbox#1{\raisebox{-\baselineskip}[0in][0in]{#1}}
      \def\midbox#1{\raisebox{-0.5\baselineskip}[0in][0in]{#1}}
 \vspace{3cm}
 \title{Assignment 10}
 \author{Adarsh Srivastava}
 \maketitle
 \newpage
 \bigskip
 %\renewcommand{\thefigure}{\theenumi}
 \renewcommand{\thetable}{\theenumi}
 The link to the solution is
 \begin{lstlisting}
  https://github.com/Adarsh1310/EE5609
 \end{lstlisting}
 \begin{abstract}
 This documents solves a problem based on fields.
 \end{abstract}
  \section{\textbf{Problem}}
Let $\mathbb{F}$ be a set which contains exactly two elements,0 and 1.Define an addition and multiplication by tables.
\begin{table}[h!]
  \begin{center}
    \label{tab:table1}
    \begin{tabular}{l|c|r}
      + & 0 & 1 \\
      \hline
      0 & 0 & 1\\
      1 & 1 & 0
    \end{tabular}
  \end{center}
\end{table}
\begin{table}[h!]
  \begin{center}
    \label{tab:table1}
    \begin{tabular}{l|c|r}
      \cdot & 0 & 1 \\
      \hline
      0 & 0 & 0\\
      1 & 0 & 1
    \end{tabular}
  \end{center}
  Verify that the set $\mathbb{F}$, together with these two operations, is a field.
\end{table}
 \section{\textbf{Solution}}
  To prove that ($\mathbb{F}$,+,$\cdot$) is a field we need to satisfy the following,
 \begin{enumerate}
\item{+ and $\cdot$ should be closed}
\begin{itemize}
\item For any a and b in $\mathbb{F}$, a+b $\in$ $\mathbb{F}$ and a $\cdot$ b $\in$ $\mathbb{F}$.For example 0+0=0 and 0$\cdot$0=0.
\end{itemize}
\item{+ and $\cdot$ should be commutative}
\begin{itemize}
\item For any a and b in $\mathbb{F}$, a+b = b+a and a $\cdot$ b = b$\cdot$ a. For example 0+1=1+0 and 0$\cdot$ 1=1$\cdot$ 0.
\end{itemize}
\item{+ and $\cdot$ should be associative}
\begin{itemize}
\item For any a and b in $\mathbb{F}$, a+(b+c) = (a+b)+c and a$\cdot$ (b $\cdot$ c) =  (a $\cdot$ b) $\cdot$ c. For example 0+(1+0)=(0+1)+0 and 0$\cdot$(1$\cdot$0)=(0$\cdot$1)$\cdot$0. 
\end{itemize}
\item{+ and $\cdot$ operations should have an identity element}
\begin{itemize}
\item If we perform a + 0 then for any value of a from $\mathbb{F}$ the result will be a itself. Hence 0 is an identity element of + operation.If we perform a $\cdot$ 1 then for any value of a from $\mathbb{F}$ the result will be a itself. Hence 1 is an identity element of $\cdot$ operation. 
\end{itemize}
\item{$\forall$ a $\in$ $\mathbb{F}$ there exists an additive inverse}
\begin{itemize}
\item For additive inverse to exist, $\forall$ a in $\mathbb{F}$ a+(-a)=0. For example. 1-1=0 and 0-0=0.
\end{itemize}
\item{$\forall$ a $\in$ $\mathbb{F}$ such that a is non zero there exists a multiplicative inverse}
\begin{itemize}
\item 
For multiplicative inverse to exist, $\forall$ a such that a is non zero in $\mathbb{F}$, a$\cdot$a^{-1}$=1.For example 1$\cdot$1^{-1}=1.
\end{itemize}
\item{+ and $\cdot$ should hold distributive property}
\begin{itemize}
\item For any a,b and c in $\mathbb{F}$ the property a$\cdot$(b+c)=a$\cdot$b+a$\cdot$c should always hold true.For example 0$\cdot$(1+1)=0$\cdot$1+0$\cdot$1.
\end{itemize}
\end{section}
\section{\textbf{Result}}
Since the above properties are satisfied we can say that ($\mathbb{F}$,+,$\cdot$) is a field.

 \end{document}