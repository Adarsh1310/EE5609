\documentclass[journal,12pt,twocolumn]{IEEEtran}
  \usepackage{setspace}
  \usepackage{gensymb}
  \usepackage{graphicx}
  \singlespacing

  \usepackage[cmex10]{amsmath}
  \usepackage{amsthm}
  \usepackage{hyperref}
  \usepackage{mathrsfs}
  \usepackage{txfonts}
  \usepackage{stfloats}
  \usepackage{bm}
  \usepackage{cite}
  \usepackage{cases}
  \usepackage{subfig}
  \usepackage{longtable}
  \usepackage{multirow}
  \usepackage{enumitem}
  \usepackage{mathtools}
  \usepackage{steinmetz}
  \usepackage{tikz}
  \usepackage{circuitikz}
  \usepackage{verbatim}
  \usepackage{tfrupee}
  \usepackage[breaklinks=true]{hyperref}
  \usepackage{tkz-euclide}
  \usetikzlibrary{calc,math}
  \usepackage{listings}
      \usepackage{color}                                            %%
      \usepackage{array}                                            %%
      \usepackage{longtable}                                        %%
      \usepackage{calc}                                             %%
      \usepackage{multirow}                                         %%
      \usepackage{hhline}                                           %%
      \usepackage{ifthen}                                           %%
      \usepackage{lscape}     
  \usepackage{multicol}
  \usepackage{chngcntr}
  \DeclareMathOperator*{\Res}{Res}
  \renewcommand\thesection{\arabic{section}}
  \renewcommand\thesubsection{\thesection.\arabic{subsection}}
  \renewcommand\thesubsubsection{\thesubsection.\arabic{subsubsection}}
  \renewcommand\thesectiondis{\arabic{section}}
  \renewcommand\thesubsectiondis{\thesectiondis.\arabic{subsection}}
  \renewcommand\thesubsubsectiondis{\thesubsectiondis.\arabic{subsubsection}}
  \hyphenation{op-tical net-works semi-conduc-tor}
  \def\inputGnumericTable{}                                 %%
  \lstset{
  %language=C,
  frame=single, 
  breaklines=true,
  columns=fullflexible
  }
  \begin{document}
  \newtheorem{theorem}{Theorem}[section]
  \newtheorem{problem}{Problem}
  \newtheorem{proposition}{Proposition}[section]
  \newtheorem{lemma}{Lemma}[section]
  \newtheorem{corollary}[theorem]{Corollary}
  \newtheorem{example}{Example}[section]
  \newtheorem{definition}[problem]{Definition}
  \newcommand{\BEQA}{\begin{eqnarray}}
  \newcommand{\EEQA}{\end{eqnarray}}
  \newcommand{\define}{\stackrel{\triangle}{=}}
  \bibliographystyle{IEEEtran}
  \providecommand{\mbf}{\mathbf}
  \providecommand{\pr}[1]{\ensuremath{\Pr\left(#1\right)}}
  \providecommand{\qfunc}[1]{\ensuremath{Q\left(#1\right)}}
  \providecommand{\sbrak}[1]{\ensuremath{{}\left[#1\right]}}
  \providecommand{\lsbrak}[1]{\ensuremath{{}\left[#1\right.}}
  \providecommand{\rsbrak}[1]{\ensuremath{{}\left.#1\right]}}
  \providecommand{\brak}[1]{\ensuremath{\left(#1\right)}}
  \providecommand{\lbrak}[1]{\ensuremath{\left(#1\right.}}
  \providecommand{\rbrak}[1]{\ensuremath{\left.#1\right)}}
  \providecommand{\cbrak}[1]{\ensuremath{\left\{#1\right\}}}
  \providecommand{\lcbrak}[1]{\ensuremath{\left\{#1\right.}}
  \providecommand{\rcbrak}[1]{\ensuremath{\left.#1\right\}}}
  \theoremstyle{remark}
  \newtheorem{rem}{Remark}
  \newcommand{\sgn}{\mathop{\mathrm{sgn}}}
  \providecommand{\abs}[1]{\left\vert#1\right\vert}
  \providecommand{\res}[1]{\Res\displaylimits_{#1}} 
  \providecommand{\norm}[1]{\left\lVert#1\right\rVert}
  %\providecommand{\norm}[1]{\lVert#1\rVert}
  \providecommand{\mtx}[1]{\mathbf{#1}}
  \providecommand{\mean}[1]{E\left[ #1 \right]}
  \providecommand{\fourier}{\overset{\mathcal{F}}{ \rightleftharpoons}}
  %\providecommand{\hilbert}{\overset{\mathcal{H}}{ \rightleftharpoons}}
  \providecommand{\system}{\overset{\mathcal{H}}{ \longleftrightarrow}}
  	%\newcommand{\solution}[2]{\textbf{Solution:}{#1}}
  \newcommand{\solution}{\noindent \textbf{Solution: }}
  \newcommand{\cosec}{\,\text{cosec}\,}
  \providecommand{\dec}[2]{\ensuremath{\overset{#1}{\underset{#2}{\gtrless}}}}
  \newcommand{\myvec}[1]{\ensuremath{\begin{pmatrix}#1\end{pmatrix}}}
  \newcommand{\mydet}[1]{\ensuremath{\begin{vmatrix}#1\end{vmatrix}}}
  \numberwithin{equation}{subsection}
  \makeatletter
  \@addtoreset{figure}{problem}
  \makeatother
  \let\StandardTheFigure\thefigure
  \let\vec\mathbf
  \renewcommand{\thefigure}{\theproblem}
  \def\putbox#1#2#3{\makebox[0in][l]{\makebox[#1][l]{}\raisebox{\baselineskip}[0in][0in]{\raisebox{#2}[0in][0in]{#3}}}}
       \def\rightbox#1{\makebox[0in][r]{#1}}
       \def\centbox#1{\makebox[0in]{#1}}
       \def\topbox#1{\raisebox{-\baselineskip}[0in][0in]{#1}}
       \def\midbox#1{\raisebox{-0.5\baselineskip}[0in][0in]{#1}}
  \vspace{3cm}
  \title{Assignment 18}
  \author{Adarsh Srivastava}
  \maketitle
  \newpage
  \bigskip
  %\renewcommand{\thefigure}{\theenumi}
  \renewcommand{\thetable}{\theenumi}
  The link to the solution is
  \begin{lstlisting}
   https://github.com/Adarsh1310/EE5609
  \end{lstlisting}
  \begin{abstract}
  This documents solves a problem based on Linear Transformation.
  \end{abstract}
   \section{\textbf{Problem}}
Describe explicitly the linear transformation T from $\mathbb{F}^2$ into $\mathbb{F}^2$ such that T({$\in$_1 })=(a,b),T({$\in$_2})=(c,d).
\section{\textbf{Solution}}
We are given a linear transformation,
\begin{align}
T:\mathbb{F}^2 \xrightarrow{} {\mathbb{F}^2}
\end{align} 
This means that T takes a two dimensional vector and returns another vector in two dimensional space after performing some linear transformation.
\begin{align}
T(\in_1)=\myvec{a\\b}\\
T(\in_2)=\myvec{c\\d}
\end{align}
 Now,let's assume $\in_1$ and $\in_2$ as linearly independent. So the linear transformation T for any vector $\vec{v}$ in two dimensional space will be,
\begin{align}
T(\vec{v})=\myvec{T(\vec{\in_1})&T(\vec{\in_2})}\myvec{k_1\\k_2}\\
=\myvec{a&c\\b&d}\myvec{k_1\\k_2}\label{range}\\
=\vec{A}\vec{v}
\end{align}
Now, there can be two cases here, transformation of linearly independent vector can be independent or it can be dependent.Considering the first case and \eqref{range} we can say that,
\begin{align}
Range(T)=\text{columnspace of} \myvec{a&c\\b&d}
\end{align}
Now, considering the case when linear transformation will be linearly dependent,
\begin{align}
Range(T)=\text{columnspace of} \myvec{a\\b}
\end{align}
Now, considering that vectors $\in_1$ and $\in_2$ itself are linearly dependent.Let $\vec{v}$=$\in_1$ + $\in_2$
\begin{align}
T(\vec{v})=T(\in_1)+T(\in_2)\\
=T(\in_1)+T(k\in_1)\\
=(k+1)T(\in_1)\\
=(k+1)\myvec{a\\b}
\end{align}
We can see from above equation that when $\in_1$ and $\in_2$ as linearly dependent then the transformation T will be along the line only.
\begin{tabular}{|c|c|} \hline
\textbf{Vectors Independent} & \textbf{Vectors Dependent}  \\ \hline
\makecell{T(\vec{v})=\myvec{a&c\\b&d}\myvec{k_1\\k_2} & T(\vec{v})=(k+1)\myvec{a\\b} } \\ \hline
\makecell{Output:& Output:
          \\On the plane & On the line \\\hline}
\end{tabular}
\end{document}

