\documentclass[journal,12pt,twocolumn]{IEEEtran}
  \usepackage{setspace}
  \usepackage{gensymb}
  \usepackage{graphicx}
  \singlespacing
 \graphicspath{ {/user/adarshsrivastava/desktop/Matrix Theory/Assignemnt_7} }
  \usepackage[cmex10]{amsmath}
  \usepackage{amsthm}
  \usepackage{hyperref}
  \usepackage{mathrsfs}
  \usepackage{txfonts}
  \usepackage{stfloats}
  \usepackage{bm}
  \usepackage{cite}
  \usepackage{cases}
  \usepackage{subfig}
  \usepackage{longtable}
  \usepackage{multirow}
  \usepackage{enumitem}
  \usepackage{mathtools}
  \usepackage{steinmetz}
  \usepackage{tikz}
  \usepackage{circuitikz}
  \usepackage{verbatim}
  \usepackage{tfrupee}
  \usepackage[breaklinks=true]{hyperref}
  \usepackage{tkz-euclide}
  \usetikzlibrary{calc,math}
  \usepackage{listings}
      \usepackage{color}                                            %%
      \usepackage{array}                                            %%
      \usepackage{longtable}                                        %%
      \usepackage{calc}                                             %%
      \usepackage{multirow}                                         %%
      \usepackage{hhline}                                           %%
      \usepackage{ifthen}                                           %%
      \usepackage{lscape}     
  \usepackage{multicol}
  \usepackage{chngcntr}
  \DeclareMathOperator*{\Res}{Res}
  \renewcommand\thesection{\arabic{section}}
  \renewcommand\thesubsection{\thesection.\arabic{subsection}}
  \renewcommand\thesubsubsection{\thesubsection.\arabic{subsubsection}}
  \renewcommand\thesectiondis{\arabic{section}}
  \renewcommand\thesubsectiondis{\thesectiondis.\arabic{subsection}}
  \renewcommand\thesubsubsectiondis{\thesubsectiondis.\arabic{subsubsection}}
  \hyphenation{op-tical net-works semi-conduc-tor}
  \def\inputGnumericTable{}                                 %%
  \lstset{
  %language=C,
  frame=single, 
  breaklines=true,
  columns=fullflexible
  }
  \begin{document}
  \newtheorem{theorem}{Theorem}[section]
  \newtheorem{problem}{Problem}
  \newtheorem{proposition}{Proposition}[section]
  \newtheorem{lemma}{Lemma}[section]
  \newtheorem{corollary}[theorem]{Corollary}
  \newtheorem{example}{Example}[section]
  \newtheorem{definition}[problem]{Definition}
  \newcommand{\BEQA}{\begin{eqnarray}}
  \newcommand{\EEQA}{\end{eqnarray}}
  \newcommand{\define}{\stackrel{\triangle}{=}}
  \bibliographystyle{IEEEtran}
  \providecommand{\mbf}{\mathbf}
  \providecommand{\pr}[1]{\ensuremath{\Pr\left(#1\right)}}
  \providecommand{\qfunc}[1]{\ensuremath{Q\left(#1\right)}}
  \providecommand{\sbrak}[1]{\ensuremath{{}\left[#1\right]}}
  \providecommand{\lsbrak}[1]{\ensuremath{{}\left[#1\right.}}
  \providecommand{\rsbrak}[1]{\ensuremath{{}\left.#1\right]}}
  \providecommand{\brak}[1]{\ensuremath{\left(#1\right)}}
  \providecommand{\lbrak}[1]{\ensuremath{\left(#1\right.}}
  \providecommand{\rbrak}[1]{\ensuremath{\left.#1\right)}}
  \providecommand{\cbrak}[1]{\ensuremath{\left\{#1\right\}}}
  \providecommand{\lcbrak}[1]{\ensuremath{\left\{#1\right.}}
  \providecommand{\rcbrak}[1]{\ensuremath{\left.#1\right\}}}
  \theoremstyle{remark}
  \newtheorem{rem}{Remark}
  \newcommand{\sgn}{\mathop{\mathrm{sgn}}}
  \providecommand{\abs}[1]{\left\vert#1\right\vert}
  \providecommand{\res}[1]{\Res\displaylimits_{#1}} 
  \providecommand{\norm}[1]{\left\lVert#1\right\rVert}
  %\providecommand{\norm}[1]{\lVert#1\rVert}
  \providecommand{\mtx}[1]{\mathbf{#1}}
  \providecommand{\mean}[1]{E\left[ #1 \right]}
  \providecommand{\fourier}{\overset{\mathcal{F}}{ \rightleftharpoons}}
  %\providecommand{\hilbert}{\overset{\mathcal{H}}{ \rightleftharpoons}}
  \providecommand{\system}{\overset{\mathcal{H}}{ \longleftrightarrow}}
  	%\newcommand{\solution}[2]{\textbf{Solution:}{#1}}
  \newcommand{\solution}{\noindent \textbf{Solution: }}
  \newcommand{\cosec}{\,\text{cosec}\,}
  \providecommand{\dec}[2]{\ensuremath{\overset{#1}{\underset{#2}{\gtrless}}}}
  \newcommand{\myvec}[1]{\ensuremath{\begin{pmatrix}#1\end{pmatrix}}}
  \newcommand{\mydet}[1]{\ensuremath{\begin{vmatrix}#1\end{vmatrix}}}
  \numberwithin{equation}{subsection}
  \makeatletter
  \@addtoreset{figure}{problem}
  \makeatother
  \let\StandardTheFigure\thefigure
  \let\vec\mathbf
  \renewcommand{\thefigure}{\theproblem}
  \def\putbox#1#2#3{\makebox[0in][l]{\makebox[#1][l]{}\raisebox{\baselineskip}[0in][0in]{\raisebox{#2}[0in][0in]{#3}}}}
       \def\rightbox#1{\makebox[0in][r]{#1}}
       \def\centbox#1{\makebox[0in]{#1}}
       \def\topbox#1{\raisebox{-\baselineskip}[0in][0in]{#1}}
       \def\midbox#1{\raisebox{-0.5\baselineskip}[0in][0in]{#1}}
  \vspace{3cm}
  \title{Assignment 15}
  \author{Adarsh Srivastava}
  \maketitle
  \newpage
  \bigskip
  %\renewcommand{\thefigure}{\theenumi}
  \renewcommand{\thetable}{\theenumi}
  The link to the solution is
  \begin{lstlisting}
   https://github.com/Adarsh1310/EE5609
  \end{lstlisting}
  \begin{abstract}
  This documents solves a problem based on vector subspaces.
  \end{abstract}
   \section{\textbf{Problem}}
 Let $\vec{W}$ be the set of all $(x_1,x_2,x_3,x_4,x_5)$ in $\mathbb{R}^5$ which satisfy
 \begin{align*}
 2x_1-x_2+\frac{4}{3}x_3-x_4=0\\x_1+\frac{2}{3}x_3-x_5=0\\9x_1-3x_2+6x_3-3x_4-3x_5=0
 \end{align*}
 Find a finite set of vectors which spans $\vec{W}$.
   \section{\textbf{Solution}}
   The above vectors can be written as,
   \begin{align}
   \vec{\alpha_1}=\myvec{2\\-1\\\frac{4}{3}\\-1\\0}\\
   \vec{\alpha_2}=\myvec{1\\0\\\frac{2}{3}\\0\\-1}\\
   \vec{\alpha_3}=\myvec{9\\-3\\6\\-3\\-3}
   \end{align}
   Vector $\aplha$ is in the subspace $\vec{W}$ of $\mathbb{R}^5$ spanned by $\vec{\alpha_1}$, $\vec{\alpha_2}$ and $ \vec{\alpha_3}$ if and only if there exist scalars $c_1$,$c_2$ as real numbers.We can see that $\vec{\alpha_3}$ is a linear combination of $\vec{\alpha_1}$ and $\vec{\alpha_2}$.
   \begin{align}
   \vec{\alpha}_3=3\myvec{2\\-1\\\frac{4}{3}\\-1\\0}+3\myvec{1\\0\\\frac{2}{3}\\0\\-1}
   \end{align}
    So,
    \begin{align}
   \vec{\alpha}=c_1\vec{\alpha_1}+c_2\vec{\alpha_2}
   \end{align}
   $\vec{W}$ consists all vector of the form,
   \begin{align}
   \vec{\alpha}=\myvec{2c_1+c_2\\-1c_1\\\frac{4}{3}c_1+\frac{2}{3}c_2\\-1c_1\\-c_2}
   \end{align}
   where $c_1$,$c_2$  are scalar constant. Alternatively,
   \begin{align}
   2x_1-x_2+\frac{4}{3}x_3-x_4=0\\
   x_1+\frac{2}{3}x_3-x_5=0
   \end{align}
   which can be written as,
   \begin{align}
  \myvec{2&-1&\frac{4}{3}&-1&0\\1&0&\frac{2}{3}&0&-1}\vec{x}=0
   \end{align}
   Now, the augmented matrix,
   \begin{align}
     \myvec{2&-1&\frac{4}{3}&-1&0&\vrule&0\\1&0&\frac{2}{3}&0&-1&\vrule& 0}\\
     \xleftrightarrow{R_2=R_2-\frac{1}{2}R_1}\myvec{2&-1&\frac{4}{3}&-1&0&\vrule&0\\0&\frac{1}{2}&0&\frac{1}{2}&-1&\vrule& 0}\\
      \xleftrightarrow{R_2=2R_2}\myvec{2&-1&\frac{4}{3}&-1&0&\vrule&0\\0&1&0&1&-2&\vrule& 0}\\
      \xleftrightarrow{R_1=R_1+R_2}\myvec{2&0&\frac{4}{3}&0&-2&\vrule&0\\0&1&0&1&-2&\vrule& 0}
\end{align}
So,
\begin{align}
2x_1+\frac{4}{3}x_3-2x_5=0\\
x_2+x_4-2x_5=0
\end{align}
Solving the equations we get,
\begin{align}
x_1=-\frac{2}{3}x_3+x_5\\
x_2=-x_4+2x_5
\end{align}
which can be written as,
   \begin{align}
   \vec{x} = \myvec{x_1\\x_2\\x_3 \\ x_4 \\ x_5} \\= \myvec{-\frac{2}{3}x_3+x_5\\-x_4+2x_5\\x_3 \\ x_4 \\ x_5}\\=x_3\myvec{-\frac{2}{3}\\0\\1\\0\\0} +x_4\myvec{0\\-1\\0\\1\\0}+x_5\myvec{1\\2\\0\\0\\1}
\end{align}
where $x_3$,$x_4$ and $x_5$ $\in$ $\mathbb{R}$.Hence, the vectors $\myvec{-\frac{2}{3}\\0\\1\\0\\0},\myvec{0\\-1\\0\\1\\0}$ and $\myvec{1\\2\\0\\0\\1}$ will span $\vec{W}$
\end{document}
