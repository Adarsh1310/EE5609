\documentclass[journal,12pt,twocolumn]{IEEEtran}
 \usepackage{setspace}
 \usepackage{gensymb}
 \usepackage{graphicx}
 \singlespacing
\graphicspath{ {/user/adarshsrivastava/desktop/Matrix Theory/Assignemnt_7} }
 \usepackage[cmex10]{amsmath}
 \usepackage{amsthm}
 \usepackage{hyperref}
 \usepackage{mathrsfs}
 \usepackage{txfonts}
 \usepackage{stfloats}
 \usepackage{bm}
 \usepackage{cite}
 \usepackage{cases}
 \usepackage{subfig}
 \usepackage{longtable}
 \usepackage{multirow}
 \usepackage{enumitem}
 \usepackage{mathtools}
 \usepackage{steinmetz}
 \usepackage{tikz}
 \usepackage{circuitikz}
 \usepackage{verbatim}
 \usepackage{tfrupee}
 \usepackage[breaklinks=true]{hyperref}
 \usepackage{tkz-euclide}
 \usetikzlibrary{calc,math}
 \usepackage{listings}
     \usepackage{color}                                            %%
     \usepackage{array}                                            %%
     \usepackage{longtable}                                        %%
     \usepackage{calc}                                             %%
     \usepackage{multirow}                                         %%
     \usepackage{hhline}                                           %%
     \usepackage{ifthen}                                           %%
     \usepackage{lscape}     
 \usepackage{multicol}
 \usepackage{chngcntr}
 \DeclareMathOperator*{\Res}{Res}
 \renewcommand\thesection{\arabic{section}}
 \renewcommand\thesubsection{\thesection.\arabic{subsection}}
 \renewcommand\thesubsubsection{\thesubsection.\arabic{subsubsection}}
 \renewcommand\thesectiondis{\arabic{section}}
 \renewcommand\thesubsectiondis{\thesectiondis.\arabic{subsection}}
 \renewcommand\thesubsubsectiondis{\thesubsectiondis.\arabic{subsubsection}}
 \hyphenation{op-tical net-works semi-conduc-tor}
 \def\inputGnumericTable{}                                 %%
 \lstset{
 %language=C,
 frame=single, 
 breaklines=true,
 columns=fullflexible
 }
 \begin{document}
 \newtheorem{theorem}{Theorem}[section]
 \newtheorem{problem}{Problem}
 \newtheorem{proposition}{Proposition}[section]
 \newtheorem{lemma}{Lemma}[section]
 \newtheorem{corollary}[theorem]{Corollary}
 \newtheorem{example}{Example}[section]
 \newtheorem{definition}[problem]{Definition}
 \newcommand{\BEQA}{\begin{eqnarray}}
 \newcommand{\EEQA}{\end{eqnarray}}
 \newcommand{\define}{\stackrel{\triangle}{=}}
 \bibliographystyle{IEEEtran}
 \providecommand{\mbf}{\mathbf}
 \providecommand{\pr}[1]{\ensuremath{\Pr\left(#1\right)}}
 \providecommand{\qfunc}[1]{\ensuremath{Q\left(#1\right)}}
 \providecommand{\sbrak}[1]{\ensuremath{{}\left[#1\right]}}
 \providecommand{\lsbrak}[1]{\ensuremath{{}\left[#1\right.}}
 \providecommand{\rsbrak}[1]{\ensuremath{{}\left.#1\right]}}
 \providecommand{\brak}[1]{\ensuremath{\left(#1\right)}}
 \providecommand{\lbrak}[1]{\ensuremath{\left(#1\right.}}
 \providecommand{\rbrak}[1]{\ensuremath{\left.#1\right)}}
 \providecommand{\cbrak}[1]{\ensuremath{\left\{#1\right\}}}
 \providecommand{\lcbrak}[1]{\ensuremath{\left\{#1\right.}}
 \providecommand{\rcbrak}[1]{\ensuremath{\left.#1\right\}}}
 \theoremstyle{remark}
 \newtheorem{rem}{Remark}
 \newcommand{\sgn}{\mathop{\mathrm{sgn}}}
 \providecommand{\abs}[1]{\left\vert#1\right\vert}
 \providecommand{\res}[1]{\Res\displaylimits_{#1}} 
 \providecommand{\norm}[1]{\left\lVert#1\right\rVert}
 %\providecommand{\norm}[1]{\lVert#1\rVert}
 \providecommand{\mtx}[1]{\mathbf{#1}}
 \providecommand{\mean}[1]{E\left[ #1 \right]}
 \providecommand{\fourier}{\overset{\mathcal{F}}{ \rightleftharpoons}}
 %\providecommand{\hilbert}{\overset{\mathcal{H}}{ \rightleftharpoons}}
 \providecommand{\system}{\overset{\mathcal{H}}{ \longleftrightarrow}}
 	%\newcommand{\solution}[2]{\textbf{Solution:}{#1}}
 \newcommand{\solution}{\noindent \textbf{Solution: }}
 \newcommand{\cosec}{\,\text{cosec}\,}
 \providecommand{\dec}[2]{\ensuremath{\overset{#1}{\underset{#2}{\gtrless}}}}
 \newcommand{\myvec}[1]{\ensuremath{\begin{pmatrix}#1\end{pmatrix}}}
 \newcommand{\mydet}[1]{\ensuremath{\begin{vmatrix}#1\end{vmatrix}}}
 \numberwithin{equation}{subsection}
 \makeatletter
 \@addtoreset{figure}{problem}
 \makeatother
 \let\StandardTheFigure\thefigure
 \let\vec\mathbf
 \renewcommand{\thefigure}{\theproblem}
 \def\putbox#1#2#3{\makebox[0in][l]{\makebox[#1][l]{}\raisebox{\baselineskip}[0in][0in]{\raisebox{#2}[0in][0in]{#3}}}}
      \def\rightbox#1{\makebox[0in][r]{#1}}
      \def\centbox#1{\makebox[0in]{#1}}
      \def\topbox#1{\raisebox{-\baselineskip}[0in][0in]{#1}}
      \def\midbox#1{\raisebox{-0.5\baselineskip}[0in][0in]{#1}}
 \vspace{3cm}
 \title{Assignment 12}
 \author{Adarsh Srivastava}
 \maketitle
 \newpage
 \bigskip
 %\renewcommand{\thefigure}{\theenumi}
 \renewcommand{\thetable}{\theenumi}
 The link to the solution is
 \begin{lstlisting}
  https://github.com/Adarsh1310/EE5609
 \end{lstlisting}
 \begin{abstract}
 This documents solves a problem based on Row Echelon form.
 \end{abstract}
  \section{\textbf{Problem}}
Suppose $\vec{R}$ and $\vec{R}^{'}$ are 2 $\times$ 3 row-reduced echelon matrices and that the system $\vec{R}$$\vec{X}$=0 and $\vec{R}^{'}$$\vec{X}$=0 have exactly the same solutions. Prove that $\vec{R}$=$\vec{R}^'$.
 \section{\textbf{Solution}}
 Since $\vec{R}$ and $\vec{R}^{'}$ are 2 $\times$ 3 row-reduced echelon matrices they can be of following three types:-
\begin{enumerate}
\item Suppose matrix $\vec{R}$ has one non-zero row then $\vec{R}$$\vec{X}$=0 will have two free variables.Since \vec{R}^{'}$$\vec{X}$=0 will have the exact same solution as 
$\vec{R}$$\vec{X}$=0, \vec{R}^{'}$$\vec{X}$=0 will also have two free variables. Thus $\vec{R}^{'}$ have one non zero row.Now let's consider a matrix $\vec{A}$ with the first row as the non-zero row $\vec{R}$ and second row as the second row of $\vec{R}^'$.
\begin{align}
\vec{R}=\myvec{1&a&b\\0&0&0}\\
\vec{R}^{'}=\myvec{1&c&d\\0&0&0}\\
  \end{align}
  Let $\vec{X}$ satisfy
  \begin{align}
  \vec{R}\vec{X}=0\label{b}\\
  \myvec{1 & \vec{a}^T}\myvec{x \\ \vec{y}} = 0\\
  x+\vec{a}^T\vec{y}=0\label{abcdef}
  \end{align}
  where 
  \begin{align}
  \vec{a}=\myvec{a \\ b}
  \end{align}
  \begin{align}
  \vec{R}^{'}\vec{X}=0\label{c}\\
   \myvec{1 & \vec{b}^T}\myvec{x \\ \vec{y}} = 0\\
   x+\vec{b}^T\vec{y}=0\label{abcde}
  \end{align}
  where 
  \begin{align}
  \vec{b}=\myvec{c \\ d}
  \end{align}
  Subtracting \eqref{abcde} from \eqref{abcdef},
  \begin{align}
  x+\vec{a}^T\vec{y}-x-\vec{b}^T\vec{y}=0\\
  (\vec{a}^T-\vec{b}^T)\vec{y}=0\label{part1}
  \end{align}
  Since $\vec{y}$ is a $2 \times 1 $ vector,
  \begin{align}
 \implies y_1 \vec{a}-y_2\vec{b} = 0
 \end{align}
 Which can be written as,
 \begin{align}
 \vec{a}=k\vec{b}
\end{align}
where,k=$\frac{\vec{y}_2}{\vec{y}_1}$ assuming \vec{y}_1 $\not$=0.\\\\\\\\\\\\\\\\\\\\Hence for \eqref{part1} to be always valid
  \begin{align}
  \vec{a}^T-\vec{b}^T=0\\
   \vec{a}^T= \vec{b}^T
   \end{align}
 Hence, $\vec{R}$=$\vec{R}^{\prime}$
 \item Let $\vec{R}$ and $\vec{R}^{'}$ have all rows as non zero.
\begin{align}\vec{R}=\myvec{1&0&b\\0&1&c} \\\vec{R}^{'}=\myvec{1&0&e\\0&1&f}\end{align}
 Let $\vec{X}$ satisfy
  \begin{align}
  \vec{R}\vec{X}=0\label{e}\\
  \vec{X}^{T}\vec{R}^{T}=0\label{fi}
  \end{align}
  Here,
  \begin{align}
  \vec{R}=\myvec{\vec{I}&\vec{a}}\label{f}\\
  \vec{a}=\myvec{b \\c}\\
  \vec{R}^{T}=\myvec{\vec{I}\\\vec{a}^T}
  \end{align}
  Let,
  \begin{align}
  \vec{X}^T=\myvec{\vec{y}^T&z}\label{g}
  \end{align}
  where z is a scalar constant. Now,substituting \eqref{g} and \eqref{f} in \eqref{fi}
  \begin{align}
  \myvec{\vec{y}^T&z}\myvec{\vec{I}\\\vec{a}^T}=0\\
  \vec{y}^T+z\vec{a}^T=0\label{49}
  \end{align}
  Now for,
  \begin{align}
  \vec{R}^{'}\vec{X}=0\label{ee}\\
  \vec{X}^{T}\vec{R}^{'}^{T}=0\label{fii}
  \end{align}
  Here,
  \begin{align}
  \vec{R}^{'}=\myvec{\vec{I}&\vec{b}}\label{ff}\\
  \vec{b}=\myvec{e\\f}
  \end{align}
  Let,
  \begin{align}
  \vec{X}^T=\myvec{\vec{y}^T&z}\label{gg}
  \end{align}
  where z is a scalar constant.Now,substituting \eqref{gg} and \eqref{ff} in \eqref{fii}
  \begin{align}
    \myvec{\vec{y}^T&z}\myvec{\vec{I}\\\vec{b}^T}=0\\
  \vec{y}^T+z\vec{b}^T=0\label{50}
  \end{align}
  Subtracting \eqref{50} from \eqref{49} 
  \begin{align}
    \vec{y}^T+z\vec{a}^T-\vec{y}^T-z\vec{b}^T=0\\
    (\vec{a}^T-\vec{b}^T)y=0\label{part2}
  \end{align}
  Hence \eqref{part2} to be always valid
  \begin{align}
  \vec{a}^T-\vec{b}^T=0\\
  \vec{a}^T=\vec{b}^T
  \end{align}
  Hence, $\vec{R}$=$\vec{R}^{\prime}$
\item Suppose matrix $\vec{R}$ have all the rows as zero then $\vec{R}$$\vec{X}$=0 will be satisfied for all values of $\vec{X}$. We know that $\vec{R}^'$$\vec{X}$=0 will have the exact same solution as 
 $\vec{R}$$\vec{X}$=0 then we can say that for all values of $\vec{X}$=0 equation $\vec{R}^{'}$$\vec{X}$=0 will be satisfied Hence, $\vec{R}^{'}$=$\vec{R}$=0.

\end{enumerate}
Since $$\vec{y}$$ is a $$2 \times 1 $$ vector,

$$(a^T-b^T)\vec{y} = 0 \implies $$y_1 \vec{a}-y_2\vec{b} = 0, or, \vec{a} = k \vec{b}, k = \frac{y_2}{y_1}$$ assuming

that $$y_1 \ne 0$$. Then proceed with your argument.

Also, consider the case when $$y_1 = 0$$.
  \end{document}