\documentclass[journal,12pt,twocolumn]{IEEEtran}
  \usepackage{setspace}
  \usepackage{gensymb}
  \usepackage{graphicx}
  \singlespacing
 \graphicspath{ {/user/adarshsrivastava/desktop/Matrix Theory/Assignemnt_7} }
  \usepackage[cmex10]{amsmath}
  \usepackage{amsthm}
  \usepackage{hyperref}
  \usepackage{mathrsfs}
  \usepackage{txfonts}
  \usepackage{stfloats}
  \usepackage{bm}
  \usepackage{cite}
  \usepackage{cases}
  \usepackage{subfig}
  \usepackage{longtable}
  \usepackage{multirow}
  \usepackage{enumitem}
  \usepackage{mathtools}
  \usepackage{steinmetz}
  \usepackage{tikz}
  \usepackage{circuitikz}
  \usepackage{verbatim}
  \usepackage{tfrupee}
  \usepackage[breaklinks=true]{hyperref}
  \usepackage{tkz-euclide}
  \usetikzlibrary{calc,math}
  \usepackage{listings}
      \usepackage{color}                                            %%
      \usepackage{array}                                            %%
      \usepackage{longtable}                                        %%
      \usepackage{calc}                                             %%
      \usepackage{multirow}                                         %%
      \usepackage{hhline}                                           %%
      \usepackage{ifthen}                                           %%
      \usepackage{lscape}     
  \usepackage{multicol}
  \usepackage{chngcntr}
  \DeclareMathOperator*{\Res}{Res}
  \renewcommand\thesection{\arabic{section}}
  \renewcommand\thesubsection{\thesection.\arabic{subsection}}
  \renewcommand\thesubsubsection{\thesubsection.\arabic{subsubsection}}
  \renewcommand\thesectiondis{\arabic{section}}
  \renewcommand\thesubsectiondis{\thesectiondis.\arabic{subsection}}
  \renewcommand\thesubsubsectiondis{\thesubsectiondis.\arabic{subsubsection}}
  \hyphenation{op-tical net-works semi-conduc-tor}
  \def\inputGnumericTable{}                                 %%
  \lstset{
  %language=C,
  frame=single, 
  breaklines=true,
  columns=fullflexible
  }
  \begin{document}
  \newtheorem{theorem}{Theorem}[section]
  \newtheorem{problem}{Problem}
  \newtheorem{proposition}{Proposition}[section]
  \newtheorem{lemma}{Lemma}[section]
  \newtheorem{corollary}[theorem]{Corollary}
  \newtheorem{example}{Example}[section]
  \newtheorem{definition}[problem]{Definition}
  \newcommand{\BEQA}{\begin{eqnarray}}
  \newcommand{\EEQA}{\end{eqnarray}}
  \newcommand{\define}{\stackrel{\triangle}{=}}
  \bibliographystyle{IEEEtran}
  \providecommand{\mbf}{\mathbf}
  \providecommand{\pr}[1]{\ensuremath{\Pr\left(#1\right)}}
  \providecommand{\qfunc}[1]{\ensuremath{Q\left(#1\right)}}
  \providecommand{\sbrak}[1]{\ensuremath{{}\left[#1\right]}}
  \providecommand{\lsbrak}[1]{\ensuremath{{}\left[#1\right.}}
  \providecommand{\rsbrak}[1]{\ensuremath{{}\left.#1\right]}}
  \providecommand{\brak}[1]{\ensuremath{\left(#1\right)}}
  \providecommand{\lbrak}[1]{\ensuremath{\left(#1\right.}}
  \providecommand{\rbrak}[1]{\ensuremath{\left.#1\right)}}
  \providecommand{\cbrak}[1]{\ensuremath{\left\{#1\right\}}}
  \providecommand{\lcbrak}[1]{\ensuremath{\left\{#1\right.}}
  \providecommand{\rcbrak}[1]{\ensuremath{\left.#1\right\}}}
  \theoremstyle{remark}
  \newtheorem{rem}{Remark}
  \newcommand{\sgn}{\mathop{\mathrm{sgn}}}
  \providecommand{\abs}[1]{\left\vert#1\right\vert}
  \providecommand{\res}[1]{\Res\displaylimits_{#1}} 
  \providecommand{\norm}[1]{\left\lVert#1\right\rVert}
  %\providecommand{\norm}[1]{\lVert#1\rVert}
  \providecommand{\mtx}[1]{\mathbf{#1}}
  \providecommand{\mean}[1]{E\left[ #1 \right]}
  \providecommand{\fourier}{\overset{\mathcal{F}}{ \rightleftharpoons}}
  %\providecommand{\hilbert}{\overset{\mathcal{H}}{ \rightleftharpoons}}
  \providecommand{\system}{\overset{\mathcal{H}}{ \longleftrightarrow}}
  	%\newcommand{\solution}[2]{\textbf{Solution:}{#1}}
  \newcommand{\solution}{\noindent \textbf{Solution: }}
  \newcommand{\cosec}{\,\text{cosec}\,}
  \providecommand{\dec}[2]{\ensuremath{\overset{#1}{\underset{#2}{\gtrless}}}}
  \newcommand{\myvec}[1]{\ensuremath{\begin{pmatrix}#1\end{pmatrix}}}
  \newcommand{\mydet}[1]{\ensuremath{\begin{vmatrix}#1\end{vmatrix}}}
  \numberwithin{equation}{subsection}
  \makeatletter
  \@addtoreset{figure}{problem}
  \makeatother
  \let\StandardTheFigure\thefigure
  \let\vec\mathbf
  \renewcommand{\thefigure}{\theproblem}
  \def\putbox#1#2#3{\makebox[0in][l]{\makebox[#1][l]{}\raisebox{\baselineskip}[0in][0in]{\raisebox{#2}[0in][0in]{#3}}}}
       \def\rightbox#1{\makebox[0in][r]{#1}}
       \def\centbox#1{\makebox[0in]{#1}}
       \def\topbox#1{\raisebox{-\baselineskip}[0in][0in]{#1}}
       \def\midbox#1{\raisebox{-0.5\baselineskip}[0in][0in]{#1}}
  \vspace{3cm}
  \title{Assignment 17}
  \author{Adarsh Srivastava}
  \maketitle
  \newpage
  \bigskip
  %\renewcommand{\thefigure}{\theenumi}
  \renewcommand{\thetable}{\theenumi}
  The link to the solution is
  \begin{lstlisting}
   https://github.com/Adarsh1310/EE5609
  \end{lstlisting}
  \begin{abstract}
  This documents solves a problem based on coordinates.
  \end{abstract}
   \section{\textbf{Problem}}
   Let $\vec{V}$ be the real vector space of all polynomial functions from $\mathbb{R}$ to $\mathbb{R}$ of degree 2 or less, i.e, the space of all functions f of the form,
   \begin{align*}
   f(x)=c_0+c_1x+c_2x^2
   \end{align*} 
   Let t be a fixed real number and define
   \begin{align*}
   g_1(x)=1,        g_2(x)=x+t,         g_3(x)=(x+t)^2
   \end{align*}
   Prove that $\beta$=\{g1,g2,g3\} is a basis for V. If 
   \begin{align*}
   f(x)=c_0+c_1x+c_2x^2
   \end{align*}
   what are the coordinates of f in the ordered basis $\beta$
   \section{\textbf{Solution}}
  . Assuming $\myvec{1\\x\\x^2}$ to be independent. Let's start by proving that $\{g_1,g_2,g_3\}$ are linearly independent,
   \begin{align}
   \vec{v}^T\myvec{g_1\\g_2\\g_3}=0\\
    \vec{v}^T\myvec{1&0&0\\t&1&0\\t^2&2t&1}\myvec{1\\x\\x^2}=0
  \end{align}
   Since $\myvec{1\\x\\x^2}$ is linearly independent so,
   \begin{align}
    \vec{v}^T\myvec{1&0&0\\t&1&0\\t^2&2t&1}=0
   \end{align} 
   Now, since the 3 $\times$ 3 matrix is linearly independent we can say tha $\vec{v}$ is zero and hence $\{g_1,g_2,g_3\}$ will be linearly independent.
  Now, to find the coordinates,
  \begin{align}
  f(x)=ag_1+bg_2+cg_3
  \end{align}
  \begin{multline}
\myvec{c_0&c_1&c_2}\myvec{1\\x\\x^2}=\myvec{a&0&0}\myvec{1\\x\\x^2}+\\\myvec{bt&b&0}\myvec{1\\x\\x^2}+\myvec{ct^2&2ct&c}\myvec{1\\x\\x^2}\end{multiline}
\begin{align}
\myvec{c_0&c_1&c_2}\myvec{1\\x\\x^2}=\myvec{a+bt+ct^2&b+2ct&c}\myvec{1\\x\\x^2}
  \end{align} 
  From above we can see that,
  \begin{align}
  c_2=c\\
  c_1=b+2ct\\
  c_0=a+bt+ct^2
  \end{align}
Solving the above equation we get,
  \begin{align}
  c=c_2\\
  b=c_1-2c_2t\\
  a=c_0-c_1t+c_2t^2
  \end{align}
  So, finally the coordinates of f in ordered basis of $\beta$,
  \begin{align}
  \myvec{c_0-c_1t+c_2t^2\\c_1-2c_2t\\c_2}
  \end{align}
\end{document}

