\documentclass[journal,12pt,twocolumn]{IEEEtran}

\usepackage{setspace}
\usepackage{gensymb}

\singlespacing


\usepackage[cmex10]{amsmath}

\usepackage{amsthm}
\usepackage{hyperref}
\usepackage{mathrsfs}
\usepackage{txfonts}
\usepackage{stfloats}
\usepackage{bm}
\usepackage{cite}
\usepackage{cases}
\usepackage{subfig}

\usepackage{longtable}
\usepackage{multirow}

\usepackage{enumitem}
\usepackage{mathtools}
\usepackage{steinmetz}
\usepackage{tikz}
\usepackage{circuitikz}
\usepackage{verbatim}
\usepackage{tfrupee}
\usepackage[breaklinks=true]{hyperref}

\usepackage{tkz-euclide}

\usetikzlibrary{calc,math}
\usepackage{listings}
    \usepackage{color}                                            %%
    \usepackage{array}                                            %%
    \usepackage{longtable}                                        %%
    \usepackage{calc}                                             %%
    \usepackage{multirow}                                         %%
    \usepackage{hhline}                                           %%
    \usepackage{ifthen}                                           %%
    \usepackage{lscape}     
\usepackage{multicol}
\usepackage{chngcntr}

\DeclareMathOperator*{\Res}{Res}

\renewcommand\thesection{\arabic{section}}
\renewcommand\thesubsection{\thesection.\arabic{subsection}}
\renewcommand\thesubsubsection{\thesubsection.\arabic{subsubsection}}

\renewcommand\thesectiondis{\arabic{section}}
\renewcommand\thesubsectiondis{\thesectiondis.\arabic{subsection}}
\renewcommand\thesubsubsectiondis{\thesubsectiondis.\arabic{subsubsection}}


\hyphenation{op-tical net-works semi-conduc-tor}
\def\inputGnumericTable{}                                 %%

\lstset{
%language=C,
frame=single, 
breaklines=true,
columns=fullflexible
}
\begin{document}


\newtheorem{theorem}{Theorem}[section]
\newtheorem{problem}{Problem}
\newtheorem{proposition}{Proposition}[section]
\newtheorem{lemma}{Lemma}[section]
\newtheorem{corollary}[theorem]{Corollary}
\newtheorem{example}{Example}[section]
\newtheorem{definition}[problem]{Definition}

\newcommand{\BEQA}{\begin{eqnarray}}
\newcommand{\EEQA}{\end{eqnarray}}
\newcommand{\define}{\stackrel{\triangle}{=}}
\bibliographystyle{IEEEtran}
\providecommand{\mbf}{\mathbf}
\providecommand{\pr}[1]{\ensuremath{\Pr\left(#1\right)}}
\providecommand{\qfunc}[1]{\ensuremath{Q\left(#1\right)}}
\providecommand{\sbrak}[1]{\ensuremath{{}\left[#1\right]}}
\providecommand{\lsbrak}[1]{\ensuremath{{}\left[#1\right.}}
\providecommand{\rsbrak}[1]{\ensuremath{{}\left.#1\right]}}
\providecommand{\brak}[1]{\ensuremath{\left(#1\right)}}
\providecommand{\lbrak}[1]{\ensuremath{\left(#1\right.}}
\providecommand{\rbrak}[1]{\ensuremath{\left.#1\right)}}
\providecommand{\cbrak}[1]{\ensuremath{\left\{#1\right\}}}
\providecommand{\lcbrak}[1]{\ensuremath{\left\{#1\right.}}
\providecommand{\rcbrak}[1]{\ensuremath{\left.#1\right\}}}
\theoremstyle{remark}
\newtheorem{rem}{Remark}
\newcommand{\sgn}{\mathop{\mathrm{sgn}}}
\providecommand{\abs}[1]{\left\vert#1\right\vert}
\providecommand{\res}[1]{\Res\displaylimits_{#1}} 
\providecommand{\norm}[1]{\left\lVert#1\right\rVert}
%\providecommand{\norm}[1]{\lVert#1\rVert}
\providecommand{\mtx}[1]{\mathbf{#1}}
\providecommand{\mean}[1]{E\left[ #1 \right]}
\providecommand{\fourier}{\overset{\mathcal{F}}{ \rightleftharpoons}}
%\providecommand{\hilbert}{\overset{\mathcal{H}}{ \rightleftharpoons}}
\providecommand{\system}{\overset{\mathcal{H}}{ \longleftrightarrow}}
	%\newcommand{\solution}[2]{\textbf{Solution:}{#1}}
\newcommand{\solution}{\noindent \textbf{Solution: }}
\newcommand{\cosec}{\,\text{cosec}\,}
\providecommand{\dec}[2]{\ensuremath{\overset{#1}{\underset{#2}{\gtrless}}}}
\newcommand{\myvec}[1]{\ensuremath{\begin{pmatrix}#1\end{pmatrix}}}
\newcommand{\mydet}[1]{\ensuremath{\begin{vmatrix}#1\end{vmatrix}}}
\numberwithin{equation}{subsection}
\makeatletter
\@addtoreset{figure}{problem}
\makeatother
\let\StandardTheFigure\thefigure
\let\vec\mathbf
\renewcommand{\thefigure}{\theproblem}
\def\putbox#1#2#3{\makebox[0in][l]{\makebox[#1][l]{}\raisebox{\baselineskip}[0in][0in]{\raisebox{#2}[0in][0in]{#3}}}}
     \def\rightbox#1{\makebox[0in][r]{#1}}
     \def\centbox#1{\makebox[0in]{#1}}
     \def\topbox#1{\raisebox{-\baselineskip}[0in][0in]{#1}}
     \def\midbox#1{\raisebox{-0.5\baselineskip}[0in][0in]{#1}}
\vspace{3cm}
\title{Challenging Problem 2}
\author{Adarsh Srivastava}
\maketitle
\newpage
\bigskip
%\renewcommand{\thefigure}{\theenumi}
\renewcommand{\thetable}{\theenumi}
The link to the solution is
\begin{lstlisting}
 https://github.com/Adarsh1310/EE5609
\end{lstlisting}
\begin{abstract}
This documents depicts that matrix multiplication is non commutative.
\end{abstract}
\section{\textbf{Problem}}

Conditions in which matrix multiplication becomes commutative are as follow:
\section{\textbf{Explanation}}

Conditions in which matrix multiplication becomes commutative are as follow:

1.Multiplication of $\vec{M}$ and $\vec{N}$ is commutative if both $\vec{M}$ and $\vec{N}$ are simultaneously diagonalisable.A matrix is said to be diagonalisable if there exist a matrix $\vec{C}$ and a matrix $\vec{D}$ such that

$\vec{M}$
=
$\bf{C^{-1}$
$\vec{M}$
$\vec{C}$

\\
\\2. Multiplication of Matrix is commutative when one or both of the matrices are $\vec{I}$
\\
\\3.If one matrix is a scalar multiple of others then also the multiplication will be commutative.
\\
4.If one of the matrix can be derived by some power of the other matrix then the multiplication of these two matrices will be commutative.
\section{Example}
\bftext{1.Diagonal Matrix}
\\
Let:-
\\
\vec{M}=\myvec{5&0\\0&7}\\
\vec{N}=\myvec{8&0\\0&9}\\

\vec{MN}=\myvec{40&0\\0&63}\\
\vec{NM}=\myvec{40&0\\0&63}\\
Hence Commutative \\
\\
\bftext{2.If one of the matrix is Identity}
\\
\\ \vec{M}=\myvec{5&0\\0&7}\\
\vec{MI}=\myvec{5&0\\0&7}\\
\vec{IM}=\myvec{5&0\\0&7}\\
Hence Commutative \\
\\
\bftext{3.One matrix scalar multiple of other:}

Let one matrix be \vec{M} and the other be 2\vec{M} then\\
\begin{align*}
\vec{N}=2\vec{M}\\
$\vec{MN}=2\vec{M}^{2}\\
$\vec{NM}=2\vec{M}^{2}\\
\end{align*}
Hence Commutative \\
\\
\bftext{4.One matrix power of other}\\
\\
Let the first matrix be \vec{M} and the other be \vec{M}^2 ,Then:
\\
\vec{M} \vec{M}^2= \vec{M}^2 \vec{M}\\
\\

Hence Commutative 


\end{document}